%%%%%%%%%%%%%%%%%%%%%%%%%%%%%%%%%%%%%%%%%%%%%%%%%%%%%%%%%%%%%%%%%%%%%%%%%%%%%
%
% elsnews-blurb.tex
%
% hamish, 19/3/2
%
% $Id: elsnews-blurb.tex,v 1.1 2002/03/19 13:38:38 hamish Exp $
%
%%%%%%%%%%%%%%%%%%%%%%%%%%%%%%%%%%%%%%%%%%%%%%%%%%%%%%%%%%%%%%%%%%%%%%%%%%%%%

Version 2 of GATE, a General Architecture for Text Engineering, was released
on March 14th 2002.
GATE is an infrastructure for developing and deploying software components 
that process human language.

GATE has been in development at the University of Sheffield since
1995 and has been used in a wide variety of research and development
projects, including Information Extraction in English,
Bulgarian, Romanian, Belgali,
Greek, Spanish, Swedish, German, Italian and French.

GATE aims to help scientists and developers in three ways:
%
\begin{enumerate}
%
\item
by specifiying an {\bf architecture}, or organisational structure,
for language processing software;
%
\item
by providing a {\bf framework}, or class library or SDK,
that implements the architecture and can be used
to embed language processing capabilities in diverse applications;
%
\item
by providing a {\bf development environment} built on top of the framework 
made up of convenient graphical tools for developing components.
\end{enumerate}
%
The architecture exploits component-based software development,
object orientation and mobile code. The framework and development
environment are written in Java and available as open-source free software.

GATE as an architecture suggests that the elements of software systems that
process natural language can usefully be broken down into various types of
component, known as resources.  Components
are reusable software chunks with well-defined interfaces; 
GATE components are specialised types of Java Bean, and come in
three flavours:
%
\begin{itemize}
%
\item 
LanguageResources (LRs) represent entities such as lexicons, corpora or
ontologies;
%
\item 
ProcessingResources (PRs) represent entities that are primarily algorithmic,
such as parsers, generators or ngram modellers;
%
\item 
VisualResources (VRs) represent visualisation and editing components that
participate in GUIs.
%
\end{itemize}
%
These definitions can be blurred in practice as necessary.

Collectively, the set of resources integrated with GATE is known as
{\bf CREOLE}: a
Collection of REusable Objects for Language Engineering. All the resources are
packaged as Java Archive (or `JAR') files, plus some XML configuration data.
The JAR and XML files are made available to GATE by putting them on a web
server, or simply placing them in the local file space.

When using GATE to develop language processing functionality for an application,
the developer uses the development environment and the framework to construct
resources of the three types. This may involve programming, or the development
of Language Resources such as grammars that are used by existing Processing
Resources, or a mixture of both. The development environment is used for
visualisation of the data structures produced and consumed during processing,
and for debugging, performance measurement and so on.
When an appropriate set of resources have been developed, they can then be
embedded in the target client application using the GATE framework. The 
framework is supplied as two JAR files.

GATE includes resources for common LE data structures and algorithms,
including documents, corpora and various annotation types, a set of
language analysis components for Information Extraction and a range of data
visualisation and editing components.
GATE supports documents in a variety of formats including XML, RTF, email,
HTML, SGML and plain text. In all cases the format is analysed and converted
into a single unified model of {\em annotation}, compatible with other
formats such as XCES, ATLAS, etc.
A family of Processing Resources for language analysis is included in the
shape of ANNIE, A Nearly-New Information Extraction system. These components
use finite state techniques to implement various tasks from tokenisation to
semantic tagging or verb phrase chunking.

Three other facilities in GATE deserve special mention:
%
\begin{itemize}
%
\item
JAPE, a Java Annotation Patterns Engine, provides regular-expression based
pattern/action rules over annotations.
%
\item
The `annotation diff' tool in the development environment implements
performance metrics such as precision and recall for comparing annotations.
%
\item
GUK, the GATE Unicode Kit, fills in some of the gaps in Java'
s support for Unicode, e.g. by adding input methods for various languages
from Urdu to Chinese.
\end{itemize}
%


%%%%%%%%%%%%%%%%%%%%%%%%%%%%%%%%%%%%%%%%%%%%%%%%%%%%%%%%%%%%%%%%%%%%%%%%%%%%
%\subsect{An Example}\label{sec:example}
%%%%%%%%%%%%%%%%%%%%%%%%%%%%%%%%%%%%%%%%%%%%%%%%%%%%%%%%%%%%%%%%%%%%%%%%%%%%
{\bf An Example}

Let's imagine that a developer called Fatima is building an email
client for Cyberdyne Systems' large corporate Intranet.
In this application she would like to have a language processing system
that automatically spots the names of people in the corporation and
transforms them into {\tt mailto} hyperlinks.

A little investigation shows that GATE's existing components can be tailored
to this purpose. Fatima starts up the development environment,
and creates a new document containing some example emails. She then
loads some processing resources that will do named-entity recognition (a
tokeniser, gazetteer and semantic tagger), and creates an application to run
these components on the document in sequence. Having processed the emails,
she can see the results in one of several viewers for annotations.

The GATE components are a decent start, but they need to be altered to deal
specially with people from Cyberdyne's personnel database. Therefore Fatima
creates new ``cyber-'' vesions of the gazetteer and semantic tagger resources,
using the ``bootstrap'' tool. This tool creates a directory structure on disk
that has some Java stub code, a Makefile and an XML configuration file.
After several hours struggling with badly written documentation, Fatima manages
to compile the stubs and create a JAR file containing the new resources. She
tells GATE the URL of these files and the
system then allows her to load them in the same way that she loaded the
built-in resources earlier on.

Fatima then creates a second copy of the email document, and uses the
annotation editing facilities to mark up the results that she would like to
see her system producing. She saves this and the version that she ran GATE on
into her Oracle datastore. From now on she can follow this routine:
%
\begin{enumerate}
\item\label{item:run}
Run her application on the email test corpus.
\item
Check the performance of the system by running the `annotation diff' tool to
compare her manual results with the system's results. This gives her both
percentage accuracy figures and a graphical display of the differences
between the machine and human outputs.
\item
Make edits to the code, pattern grammars or gazetteer lists in her resources,
and recompile where necessary.
\item
Tell GATE to re-initialise the resources.
\item
Go to \ref{item:run}.
\end{enumerate}
%
To make the alterations that she requires, Fatima re-implements the ANNIE
gazetteer so that it regenerates itself from the local personnel data.
She then alters the pattern grammar in the semantic tagger to prioritise
recognition of names from that source. 

Eventually the system is running nicely.
Now Fatima stops using the GATE development
environment and works instead on embedding the new components in her email
application. This application is written in Java, so embedding is very
easy -- the
code to talk to GATE takes up only around 150 lines of the eventual
application, most of which is just copied from the example in the
{\tt StandAloneAnnie} class on the web site.

Because Fatima is worried about Cyberdyne's unethical policy of developing
Skynet to help the large corporates of the West strengthen their
strangle-hold over the World, she wants to get a job as an academic instead
(so that her conscience will only have to cope with the torture of students,
as opposed to humanity). She takes the accuracy measures that she has
attained for her system and writes a paper for the Journal of Nasturtium
Logarithm Encitement describing the approach used and the results obtained.
Because she used GATE for development, she can cite the repeatability of her
experiments and offer access to example binary versions of her software by 
putting them on an external web server.
