%%%%%%%%%%%%%%%%%%%%%%%%%%%%%%%%%%%%%%%%%%%%%%%%%%%%%%%%%%%%%%%%%%%%%%%%%%%%%
%
% postag.tex
%
% diana, October 2002
%
% $Id: postag.tex,v 1.4 2006/04/06 11:18:24 diana Exp $
%
%%%%%%%%%%%%%%%%%%%%%%%%%%%%%%%%%%%%%%%%%%%%%%%%%%%%%%%%%%%%%%%%%%%%%%%%%%%%%


%%%%%%%%%%%%%%%%%%%%%%%%%%%%%%%%%%%%%%%%%%%%%%%%%%%%%%%%%%%%%%%%%%%%%%%%%%%%%
\chapt[chap:postags]{Part-of-Speech Tags used in the Hepple Tagger}
\markboth{Part-of-Speech Tags used in the Hepple Tagger}{Part-of-Speech Tags used in the Hepple Tagger}
%%%%%%%%%%%%%%%%%%%%%%%%%%%%%%%%%%%%%%%%%%%%%%%%%%%%%%%%%%%%%%%%%%%%%%%%%%%%%
CC - coordinating conjunction: `and', `but', `nor', `or', `yet', plus, minus, less, times (multiplication), over (division). Also `for' (because) and `so' (i.e., `so that').\\
CD - cardinal number\\
DT - determiner: Articles including `a', `an', `every', `no', `the', `another', `any', `some', `those'.\\
EX - existential `there': Unstressed `there' that triggers inversion of the inflected verb and the logical subject; `There was a party in progress'.\\
FW - foreign word\\
IN - preposition or subordinating conjunction\\
JJ - adjective: Hyphenated compounds that are used as modifiers; happy-go-lucky.\\
JJR - adjective - comparative: Adjectives with the comparative ending `-er' and a comparative meaning. Sometimes `more' and `less'.\\
JJS - adjective - superlative: Adjectives with the superlative ending `-est'
(and `worst'). Sometimes `most' and `least'.\\
JJSS - -unknown-, but probably a variant of JJS\\
-LRB- - -unknown-\\
LS - list item marker: Numbers and letters used as identifiers of items in a list.\\
MD - modal: All verbs that don't take an `-s' ending in the third person singular present: `can', `could', `dare', `may', `might', `must', `ought', `shall', `should', `will', `would'.\\
NN - noun - singular or mass\\
NNP - proper noun - singular: All words in names usually are capitalized but titles might not be.\\
NNPS - proper noun - plural: All words in names usually are capitalized but titles might not be.\\
NNS - noun - plural\\
NP - proper noun - singular\\
NPS - proper noun - plural\\
PDT - predeterminer: Determiner like elements preceding an article or possessive pronoun; `all/PDT his marbles', `quite/PDT a mess'.\\
POS - possessive ending: Nouns ending in `'s' or `'{}'.\\
PP - personal pronoun\\
PRPR\$ - unknown-, but probably possessive pronoun\\
PRP - unknown-, but probably possessive pronoun\\
PRP\$ - unknown, but probably possessive pronoun,such as `my', `your', `his', `his', `its', `one's', `our', and `their'.\\
RB - adverb: most words ending in `-ly'. Also `quite', `too', `very', `enough', `indeed', `not', `-n't', and `never'.\\
RBR - adverb - comparative: adverbs ending with `-er' with a comparative meaning.\\
RBS - adverb - superlative\\
RP - particle: Mostly monosyllabic words that also double as directional adverbs.\\
STAART - start state marker (used internally)\\
SYM - symbol: technical symbols or expressions that aren't English words.\\
TO - literal ``to''\\
UH - interjection: Such as `my', `oh', `please', `uh', `well', `yes'.\\
VBD - verb - past tense: includes conditional form of the verb `to be'; `If I were/VBD rich...'.\\
VBG - verb - gerund or present participle\\
VBN - verb - past participle\\
VBP - verb - non-3rd person singular present\\
VB - verb - base form: subsumes imperatives, infinitives and subjunctives.\\
VBZ - verb - 3rd person singular present\\
WDT - `wh'-determiner\\
WP\$ - possessive `wh'-pronoun: includes `whose'\\
WP - `wh'-pronoun: includes `what', `who', and `whom'.\\
WRB - `wh'-adverb: includes `how', `where', `why'. Includes `when' when used in a temporal sense.\\
\\
:: - literal colon\\
, - literal comma\\
\$ - literal dollar sign\\
- - literal double-dash\\
`` - literal double quotes\\
\'{} - literal grave\\
( - literal left parenthesis\\
. - literal period\\
\# - literal pound sign\\
) - literal right parenthesis\\
' - literal single quote or apostrophe\\




