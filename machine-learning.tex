
%%%%%%%%%%%%%%%%%%%%%%%%%%%%%%%%%%%%%%%%%%%%%%%%%%%%%%%%%%%%%%%%%%%%%%%%%%%%%
\chapt[chap:ml]{Machine Learning}
\markboth{Machine Learning}{Machine Learning}
%%%%%%%%%%%%%%%%%%%%%%%%%%%%%%%%%%%%%%%%%%%%%%%%%%%%%%%%%%%%%%%%%%%%%%%%%%%%%

The machine learning technology in GATE is the \textbf{Learning
Framework} plugin. This is available in the plugin manager.

A few words of introduction will be given in this section. However,
much more extensive documentation can be found here, including a step
by step tutorial:

\url{https://gatenlp.github.io/gateplugin-LearningFramework/}

\sect[sec:ml:mlingate]{Brief introduction to machine learning in GATE}

There are two main types of ML; supervised learning and unsupervised
learning. Classification is a particular example of supervised
learning, in which the set of training examples is split into multiple
subsets (classes) and the algorithm attempts to distribute new
examples into the existing classes. This is the type of ML that is
used in GATE.

An ML algorithm `learns' about a phenomenon by looking at a set of occurrences
of that phenomenon that are used as examples. Based on these, a model is built
that can be used to predict characteristics of future (unseen) examples of the
phenomenon.

An ML implementation has two modes of functioning: training and application.  The
training phase consists of building a model (e.g. a statistical model, a decision
tree, a rule set, etc.) from a dataset of already classified instances.  During
application, the model built during training is used to classify new instances.

The Learning Framework offers two main task types:

\begin{itemize}
  
\item {\bf Text classification}
classifies text into pre-defined categories. The process can be
equally well applied at the document, sentence or token level. Typical
examples of text classification might be document classification,
opinionated sentence recognition, POS tagging of tokens and word sense
disambiguation.

\item {\bf Chunk recognition} assigns a label or labels to
chunks of text. These may be classified into one or several types (for
example, Persons and Organizations may be done
simultaneously). Examples of chunk recognition include named entity
recognition (and more generally, information extraction), NP chunking
and word segmentation.

\end{itemize}

Typically, the three types of NLP learning use different linguistic features and
feature representations. For example, it has been recognised that for text
classification the so-called $tf-idf$ representation of n-grams is very effective
(e.g. with SVM). For chunk recognition, identifying the start token and the end
token of the chunk by using the linguistic features of the token itself and the
surrounding tokens is effective and efficient.

Relation learning can be implemented using classification by first
learning the entities involved in the relationship, then creating a
new instance annotation for every possible pair, then classifying the
pairs.

Some important concepts to be familiar with are:

\begin{itemize}
  
\item \textbf{instance}: an example of the studied phenomenon. An ML
algorithm learns a model from a set of known instances, called a
(training) dataset. It can then apply the learned model to another
(application) dataset. In order to use ML in GATE, annotations are
used to indicate the instances. For example, for chunking
tasks, \textit{tokens} are normally used, and are classified into the
beginning, inside or outside of the entity. For classification of
tweets into positive or negative, the instance annotation might be
``tweet''.

\item \textbf{attribute}: a characteristic of the instances. Each instance is
defined by the values of its attributes. The set of possible
attributes is well defined and is the same for all instances in the
training and application datasets. `Feature' is also often used, and
should not be confused with GATE annotation features. An attribute
must be the value of a named feature of a particular annotation type,
which might be colocated with the instance, or be before or after
it.

\item \textbf{class}: The classification to be learned, such as ``positive'' or ``negative'' for a review, or a score, or whether an entity is a person or organization. ML is used to find the value of this attribute
  in the application dataset. Any attribute referring to the current
  instance can be marked as class attribute. The exception is for
  chunking tasks, where class is specified as a type, and turning that
  into a classification task is done for you behind the scenes.

\end{itemize}

In the usual case, in a GATE corpus pipeline application, documents
are processed one at a time, and each PR is applied in turn to the
document, processing it fully, before moving on to the next
document. Machine learning PRs break from this rule. ML training
algorithms typically run as a batch process over a training set, and
require all the data to be fully prepared and passed to the algorithm
in one go. This means that in training (or evaluation) mode, the PR
will wait for all the documents to be processed and will then run as a
single operation at the end. Therefore, learning PRs need to be
positioned
\textit{last} in the pipeline. In application mode, the situation
is slightly different, since the ML model has already been created,
and the PR only applies it to the data, so the application PR can be
positioned anywhere in the pipeline.
