%%%%%%%%%%%%%%%%%%%%%%%%%%%%%%%%%%%%%%%%%%%%%%%%%%%%%%%%%%%%%%%%%%%%%%%%%%%%%
%
% corpora.tex
%
% hamish, 25/8/1
%
% $Id: corpora.tex,v 1.23 2006/07/19 13:27:50 ian Exp $
%
%%%%%%%%%%%%%%%%%%%%%%%%%%%%%%%%%%%%%%%%%%%%%%%%%%%%%%%%%%%%%%%%%%%%%%%%%%%%%


%%%%%%%%%%%%%%%%%%%%%%%%%%%%%%%%%%%%%%%%%%%%%%%%%%%%%%%%%%%%%%%%%%%%%%%%%%%%%
\chapt[chap:corpora]{Language Resources: Corpora, Documents and Annotations}
\markboth{Language Resources: Corpora, Documents and Annotations}{Language Resources: Corpora, Documents and Annotations}
%\chapt{Corpora, Documents and Annotations}\label{chap:corpora}
%%%%%%%%%%%%%%%%%%%%%%%%%%%%%%%%%%%%%%%%%%%%%%%%%%%%%%%%%%%%%%%%%%%%%%%%%%%%%
\nnormalsize
%%%% qqqqqqqqqqqqqqqqqqqqqqqqq %%%%
\ifprintedbook \else
\begin{quote}
Sometimes in life you've got to dance like nobody's watching.
\\
\ldots

I think they should introduce `sleeping' to the Olympics. It would be an 
excellent field event, in which the `athletes' (for want of a better word)
all lay down in beds, just beyond where the javelins land,
and the first one to fall asleep and not wake up for three hours
would win gold. I, for one, would be interested in seeing what kind of
personality would be suited to sleeping in a competitive environment.
\\
\ldots

Life is a mystery to be lived, not a problem to be solved.

{\it Round Ireland with a Fridge}, Tony Hawks, 1998 (pp. 119, 147, 179).
\end{quote}
\fi
%%%% qqqqqqqqqqqqqqqqqqqqqqqqq %%%%

This \chapthing\ documents GATE's model of corpora, documents and annotations
on documents. Section \ref{sec:corpora:features} describes the simple
attribute/value data model that corpora, documents and annotations all share. Section
\ref{sec:corpora:corpora}, Section \ref{sec:corpora:documents} and Section
\ref{sec:corpora:dags} describe corpora, documents and annotations on documents
respectively. Section \ref{sec:corpora:formats} describes GATE's support for
diverse document formats, and Section \ref{sec:corpora:xml} describes
facilities for XML input/output.


%%%%%%%%%%%%%%%%%%%%%%%%%%%%%%%%%%%%%%%%%%%%%%%%%%%%%%%%%%%%%%%%%%%%%%%%%%%%%
\sect[sec:corpora:features]{Features: Simple Attribute/Value Data}
%%%%%%%%%%%%%%%%%%%%%%%%%%%%%%%%%%%%%%%%%%%%%%%%%%%%%%%%%%%%%%%%%%%%%%%%%%%%%

GATE has a single model for information that describes documents, collections
of documents (corpora), and annotations on documents, based on attribute/value
pairs. Attribute names are strings; values can be any Java object. The API
for accessing this feature data is Java's {\tt Map} interface (part of the
Collections API).


%%%%%%%%%%%%%%%%%%%%%%%%%%%%%%%%%%%%%%%%%%%%%%%%%%%%%%%%%%%%%%%%%%%%%%%%%%%%%
\sect[sec:corpora:corpora]{Corpora: Sets of Documents plus Features}
%%%%%%%%%%%%%%%%%%%%%%%%%%%%%%%%%%%%%%%%%%%%%%%%%%%%%%%%%%%%%%%%%%%%%%%%%%%%%

A Corpus in GATE is a Java Set whose members are Documents.
Both Corpora and Documents are types of LanguageResource (LR); all
LRs have a FeatureMap (a Java Map) associated with them that
stored attribute/value information about the resource. FeatureMaps are also
used to associate arbitrary information with ranges of documents (e.g. pieces
of text) via the annotation model (see below).

Documents have a DocumentContent which is
a text at present (future versions may add support for audiovisual content)
and one or more AnnotationSets which are Java Sets.


%%%%%%%%%%%%%%%%%%%%%%%%%%%%%%%%%%%%%%%%%%%%%%%%%%%%%%%%%%%%%%%%%%%%%%%%%%%%%
\sect[sec:corpora:documents]{Documents: Content plus Annotations plus Features}
%%%%%%%%%%%%%%%%%%%%%%%%%%%%%%%%%%%%%%%%%%%%%%%%%%%%%%%%%%%%%%%%%%%%%%%%%%%%%

Documents are modelled as content plus annotations (see Section
\ref{sec:corpora:dags}) plus features (see Section \ref{sec:corpora:features}).
The content of a document can be any subclass of {\tt DocumentContent}.


%%%%%%%%%%%%%%%%%%%%%%%%%%%%%%%%%%%%%%%%%%%%%%%%%%%%%%%%%%%%%%%%%%%%%%%%%%%%%
\sect[sec:corpora:dags]{Annotations: Directed Acyclic Graphs}
%%%%%%%%%%%%%%%%%%%%%%%%%%%%%%%%%%%%%%%%%%%%%%%%%%%%%%%%%%%%%%%%%%%%%%%%%%%%%

Annotations are organised in graphs, which are modelled as Java sets of 
Annotation.  Annotations may be considered as the arcs in the graph; 
they have a start Node and an end Node, an ID, a type and a
FeatureMap. Nodes have pointers into the sources document, e.g. character
offsets.


%%%%%%%%%%%%%%%%%%%%%%%%%%%%%%%%%%%%%%%%%%%%%%%%%%%%%%%%%%%%%%%%%%%%%%%%%%%%%
\subsect[sec:corpora:schemas]{Annotation Schemas}
%%%%%%%%%%%%%%%%%%%%%%%%%%%%%%%%%%%%%%%%%%%%%%%%%%%%%%%%%%%%%%%%%%%%%%%%%%%%%

Annotation schemas provide a means to define types of annotations in GATE.
GATE uses the XML Schema language supported by W3C for these definitions.
When using GATE Developer to
create/edit annotations, a component is available
({\tt gate.gui.SchemaAnnotationEditor}) which is
driven by an annotation schema file. This component will constrain the data
entry process to ensure that only annotations that correspond to a particular
schema are created. (Another component allows unrestricted annotations to be
created.)

Schemas are resources just like other GATE components. Below we give some
examples of such schemas. Section \ref{sec:developer:schemaannotationeditor}
describes how to create new schemas.  Note that each schema file defines a
single annotation type, however it is possible to use \emph{include}
definitions in a schema to refer to other schemas in order to load a whole set
of schemas as a group.  The default schemas for ANNIE annotation types (defined
in \verb!resources/schema! in the ANNIE plugin) give an example of this
technique.

\small
\subsubsection*{Date Schema}
\begin{small}
\begin{verbatim}
<?xml version="1.0"?>
<schema
xmlns="http://www.w3.org/2000/10/XMLSchema">
 <!-- XSchema deffinition for Date-->
  <element name="Date">
    <complexType>
      <attribute name="kind"  use="optional">
        <simpleType>
          <restriction base="string">
            <enumeration value="date"/>
            <enumeration value="time"/>
            <enumeration value="dateTime"/>
          </restriction>
        </simpleType>
    </attribute>
  </complexType>
 </element>
</schema>
\end{verbatim}
\end{small}

\subsubsection*{Person Schema}
\begin{small}
\begin{verbatim}
<?xml version="1.0"?>
<schema
xmlns="http://www.w3.org/2000/10/XMLSchema">
    <!-- XSchema definition for Person-->
    <element name="Person" />
</schema>
\end{verbatim}
\end{small}

\subsubsection*{Address Schema}
\begin{small}
\begin{verbatim}
<?xml version="1.0"?> <schema
xmlns="http://www.w3.org/2000/10/XMLSchema">
    <!-- XSchema definition for Address-->
    <element name="Address">
      <complexType>
        <attribute name="kind"  use="optional">
          <simpleType>
            <restriction base="string">
              <enumeration value="email"/>
              <enumeration value="url"/>
              <enumeration value="phone"/>
              <enumeration value="ip"/>
              <enumeration value="street"/>
              <enumeration value="postcode"/>
              <enumeration value="country"/>
              <enumeration value="complete"/>
            </restriction>
        </simpleType>
    </attribute>
  </complexType>
</element>
</schema>
\end{verbatim}
\end{small}
\nnormalsize


%%%%%%%%%%%%%%%%%%%%%%%%%%%%%%%%%%%%%%%%%%%%%%%%%%%%%%%%%%%%%%%%%%%%%%%%%%%%%
\subsect[sec:corpora:ann-examples]{Examples of Annotated Documents}
%%%%%%%%%%%%%%%%%%%%%%%%%%%%%%%%%%%%%%%%%%%%%%%%%%%%%%%%%%%%%%%%%%%%%%%%%%%%%

This section shows some simple examples of annotated documents.

This material is adapted from \cite{Gri96b}, the TIPSTER Architecture Design
document upon which GATE version 1 was based. Version 2 has a similar model,
although annotations are now graphs, and instead of multiple spans per
annotation each annotation now has a single start/end node pair. The current
model is largely compatible with \cite{Bir99}, and roughly isomorphic with
{\tt{}"{}}stand-off markup{\tt{}"{}} as latterly adopted by the SGML/XML
community.

Each example is shown in the form of a table. At the top of the table is the
document being annotated; immediately below the line with the document is a
ruler showing the position (byte offset) of each character 
(see \htlink{http://www.itl.nist.gov/iaui/894.02/related\string _projects/tipster/}{TIPSTER Architecture Design Document}).

Underneath this appear the annotations, one annotation per line. 
For each annotation is shown its Id, Type, Span (start/end offsets derived 
from the start/end nodes), and Features. Integers are used as the annotation 
Ids.  The features are shown in the form name = value.

The first example shows a single sentence and the result of three
annotation procedures: tokenization with part-of-speech assignment, name
recognition, and sentence boundary recognition. Each token has a single
feature, its part of speech (pos), using the tag set from the University of
Pennsylvania Tree Bank; each name also has a single feature, indicating the
type of name: person, company, etc.


\begin{table}
\begin{center}
\begin{tabular}{|l|l|l|l|l|}
\hline
 \multicolumn{5}{|c|}{\textbf{Text}}\\
\hline
\multicolumn{5}{|c|}
 {\texttt{Cyndi savored the soup.}}\\
\hline
\multicolumn{5}{|c|}
 {\texttt{\^{}0...\^{}5...\^{}10..\^{}15..\^{}20}}\\
\hline
\multicolumn{5}{|c|}
  {\textbf{Annotations}}\\
\hline
  Id & Type & SpanStart & Span End & Features\\
\hline
  1 & token & 0 & 5 & pos=NP\\
\hline
  2 & token & 6 & 13 & pos=VBD\\
\hline
  3 & token & 14 & 17 & pos=DT\\
\hline
  4 & token & 18 & 22 & pos=NN\\
\hline
  5 & token & 22 & 23 & \\
\hline
  6 & name & 0 & 5 & name\_type=person\\
\hline
  7 & sentence & 0 & 23 & \\
\hline
\end{tabular}
\caption{Result of annotation on a single sentence}
\label{table:annotation1}
\end{center}
\end{table}


Annotations will typically be organized to describe a hierarchical
decomposition of a text. A simple illustration would be the decomposition of a
sentence into tokens. A more complex case would be a full syntactic analysis,
in which a sentence is decomposed into a noun phrase and a verb phrase, a verb
phrase into a verb and its complement, etc. down to the level of individual
tokens. Such decompositions can be represented by annotations on nested sets of
spans. Both of these are illustrated in the second example, which is an
elaboration of our first example to include parse information. Each
non-terminal node in the parse tree is represented by an annotation of type
parse.

\begin{table}
\begin{center}
\begin{tabular}{|l|l|l|l|l|}
\hline
 \multicolumn{5}{|c|}{\textbf{Text}}\\
\hline
\multicolumn{5}{|c|}
  {\texttt{Cyndi savored the soup.}}\\
\hline
\multicolumn{5}{|c|}
  {\texttt{\^{}0...\^{}5...\^{}10..\^{}15..\^{}20}}\\
\hline
\multicolumn{5}{|c|}
  {\textbf{Annotations}}\\
\hline
  Id & Type & SpanStart & Span End & Features\\
\hline
  1 & token & 0 & 5 & pos=NP\\
\hline
  2 & token & 6 & 13 & pos=VBD\\
\hline
  3 & token & 14 & 17 & pos=DT\\
\hline
  4 & token & 18 & 22 & pos=NN\\
\hline
  5 & token & 22 & 23 & \\
\hline
  6 & name & 0 & 5 & name\_type=person\\
\hline
  7 & sentence & 0 & 23 & constituents=[1],[2],[3].[4],[5]\\
\hline
\end{tabular}
\caption{Result of annotations including parse information}
\label{table:annotation2}
\end{center}
\end{table}

In most cases, the hierarchical structure could be recovered from the
spans. However, it may be desirable to record this structure directly through a
constituents feature whose value is a sequence of annotations representing
the immediate constituents of the initial annotation. For the annotations of
type parse, the constituents are either non-terminals (other annotations in the
parse group) or tokens. For the sentence annotation, the constituents feature
points to the constituent tokens. A reference to another annotation is
represented in the table as {\tt{}"{}}\mbox{$[$} Annotation Id\mbox{$]$}{\tt{}"{}}; for example, {\tt{}"{}}\mbox{$[$}3\mbox{$]$}{\tt{}"{}} represents a
reference to annotation 3. Where the value of an feature is a sequence of items, these items are separated by commas. No special operations are provided
in the current architecture for manipulating constituents. At a less esoteric
level, annotations can be used to record the overall structure of documents,
including in particular documents which have structured headers, as is shown in
the third example (Table \ref{table:annotation3}).


\begin{table}
\begin{center}
\begin{tabular}{|l|l|l|l|l|}
\hline
\multicolumn{5}{|c|}
  {\textbf{Text}}\\
\hline
\multicolumn{5}{|c|}
  {\texttt{To: All Barnyard Animals}}\\
\hline
\multicolumn{5}{|c|}
  {\texttt{\^{}0...\^{}5...\^{}10..\^{}15..\^{}20.}}\\
\hline
\multicolumn{5}{|c|}
  {\texttt{From: Chicken Little}}\\
\hline
\multicolumn{5}{|c|}
  {\texttt{\^{}25..\^{}30..\^{}35..\^{}40..}}\\
\hline
\multicolumn{5}{|c|}
  {\texttt{Date: November 10,1194}}\\
\hline
\multicolumn{5}{|c|}
  {\texttt{...\^{}50..\^{}55..\^{}60..\^{}65.}}\\
\hline
\multicolumn{5}{|c|}
  {\texttt{Subject: Descending Firmament}}\\
\hline
\multicolumn{5}{|c|}
  {\texttt{.\^{}70..\^{}75..\^{}80..\^{}85..\^{}90..\^{}95}}\\
\hline
\multicolumn{5}{|c|}
  {\texttt{Priority: Urgent}}\\
\hline
\multicolumn{5}{|c|}
  {\texttt{.\^{}100.\^{}105.\^{}110.}}\\
\hline
\multicolumn{5}{|c|}
  {\texttt{The sky is falling. The sky is falling.}}\\
\hline
\multicolumn{5}{|c|}
  {\texttt{....\^{}120.\^{}125.\^{}130.\^{}135.\^{}140.\^{}145.\^{}150.}}\\
\hline
\multicolumn{5}{|c|}
  {\textbf{Annotations}}\\
\hline
  Id & Type & SpanStart & Span End & Features\\
\hline                         
  1 & Addressee & 4 & 24 & \\
\hline
  2 & Source & 31 & 45 & \\
\hline
  3 & Date & 53 & 69 & ddmmyy=101194\\
\hline
  4 & Subject& 78 & 98 & \\
\hline
  5 & Priority & 109 & 115 & \\
\hline
  6 & Body & 116 & 155 & \\
\hline
  7 & Sentence & 116 & 135 &\\
\hline
  8 & Sentence & 136 & 155 & \\
\hline
\end{tabular}
\caption{Annotation showing overall document structure}
\label{table:annotation3}
\end{center}
\end{table}


If the Addressee, Source, ... annotations are recorded when the
document is indexed for retrieval, it will be possible to perform retrieval
selectively on information in particular fields. Our final example (Table \ref{table:annotation4})
involves an annotation which effectively modifies the document. The current
architecture does not make any specific provision for the modification of the
original text. However, some allowance must be made for processes such as
spelling correction. This information will be recorded as a correction
feature on token annotations and possibly on name
annotations:

\begin{table}
\begin{center}
\begin{tabular}{|l|l|l|l|l|}
\hline
\multicolumn{5}{|c|}
  {\textbf{Text}}\\
\hline
\multicolumn{5}{|c|}
  {\texttt{Topster tackles 2 terrorbytes.}}\\
\hline
\multicolumn{5}{|c|}
  {\texttt{\^{}0...\^{}5...\^{}10..\^{}15..\^{}20..\^{}25..}}\\
\hline
\multicolumn{5}{|c|}
  {\textbf{Annotations}}\\
\hline

  Id & Type & SpanStart & Span End & Features\\
\hline                         
  1 & token & 0 & 7 & pos=NP correction=TIPSTER\\
\hline                         
  2 & token & 8 & 15 & pos=VBZ\\
\hline                         
  3 & token & 16 & 17 & pos=CD\\
\hline                         
  4 & token & 18 & 29 & pos=NNS correction=terabytes\\
\hline                         
  5 & token & 29 & 30 & \\
\hline

\end{tabular}
\caption{Annotation modifying the document}
\label{table:annotation4}
\end{center}
\end{table}

%%%%%%%%%%%%%%%%%%%%%%%%%%%%%%%%%%%%%%%%%%%%%%%%%%%%%%%%%%%%%%%%%%%%%%%%%%%%%
\subsect[sec:corpora:viewing]{Creating, Viewing and Editing Diverse Annotation Types}
%%%%%%%%%%%%%%%%%%%%%%%%%%%%%%%%%%%%%%%%%%%%%%%%%%%%%%%%%%%%%%%%%%%%%%%%%%%%%

Note that annotation types should consist of a single word with no spaces.
Otherwise they may not be recognised by other components such as JAPE
transducers, and may create problems when annotations are saved as inline (`Save
Preserving Format' in the context menu).

To view and edit annotation types, see Section \ref{sec:developer:annotations}.
To add annotations of a new type, see Section \ref{sec:developer:edit}.
To add a new annotation schema, see Section
\ref{sec:developer:schemaannotationeditor}.

%%%%%%%%%%%%%%%%%%%%%%%%%%%%%%%%%%%%%%%%%%%
%\sect{Ontology-based Corpus Annotation Tool}\label{sec:corpora:ocat}

%The Ontology-based Corpus Annotation Tool (OCAT) is a GATE plugin, which uses
%one or more ontologies for annotation. Version 1 of OCAT supports only
%annotation with information about the ontology class. Future versions will
%support annotation with instance information and properties. Details of its
%functionality can be found in Section \ref{sec:ontologies:ocat}.




%%%%%%%%%%%%%%%%%%%%%%%%%%%%%%%%%%%%%%%%%%%%%%%%%%%%%%%%%%%%%%%%%%%%%%%%%%%%%
\sect[sec:corpora:formats]{Document Formats}
%%%%%%%%%%%%%%%%%%%%%%%%%%%%%%%%%%%%%%%%%%%%%%%%%%%%%%%%%%%%%%%%%%%%%%%%%%%%%

The following document formats are supported by GATE by default:
\begin{itemize}
\item
Plain Text
\item
HTML
\item
SGML
\item
XML
\item
RTF
\item
Email
\item
PDF (some documents)
\item
Microsoft Office (some formats)
\item
OpenOffice (some formats)
\item
UIMA CAS XML format
\item
CoNLL/IOB
\end{itemize}

Additional formats are provided by plugins -- you must load the relevant
plugin before attempting to parse these document types
\begin{itemize}
\item Twitter JSON (in the {\tt Twitter} plugin, see
  section~\ref{sec:social:twitter:format})
\item DataSift JSON, a common format for social media data from
  \htlinkplain{http://datasift.com} (in the {\tt Format\_DataSift} plugin, see
  section~\ref{sec:creole:datasift})
\item FastInfoset, a compressed binary encoding of GATE XML (in the
  {\tt Format\_FastInfoset} plugin, see section~\ref{sec:creole:fastinfoset})
\item MediaWiki markup, as used by Wikipedia and many other public wiki sites
  (in the {\tt Format\_MediaWiki} plugin, see
  section~\ref{sec:creole:mediawiki})
\item The formats used by PubMed and the Cochrane collaboration for biomedical
  literature (in the {\tt Format\_PubMed} plugin, see
  section~\ref{sec:creole:pubmed})
\item CSV files containing one column of text data and optionally additional
  columns of metadata (in the {\tt Format\_CSV} plugin, see
  section~\ref{sec:creole:csv})
\end{itemize}

By default GATE will try and identify the type of the document, then strip
and convert any markup into GATE's annotation format. To disable this
process, set the {\tt markupAware} parameter on the document to {\tt false}.

When reading a document of one of these types, GATE extracts the text
between tags (where such exist)
and create a GATE annotation filled as follows:

The name of the tag will constitute the annotation's type, all the
tags attributes will materialize in the annotation's features and
the annotation will span over the text covered by the tag. A few
exceptions of this rule apply for the RTF, Email and Plain Text
formats, which will be described later in the input section of
these formats.

The text between tags is extracted and appended to the GATE
document's content and all annotations created from tags will be
placed into a GATE annotation set named `Original markups'.

{\em Example:}

If the markup is like this:

\small
\begin{small}
\begin{verbatim}
<aTagName attrib1="value1" attrib2="value2" attrib3="value3"> A
piece of text</aTagName>
\end{verbatim}
\end{small}
\nnormalsize

then the annotation created by GATE will look like:

\small
\begin{small}
\begin{verbatim}
annotation.type = "aTagName";
annotation.fm = {attrib1=value1;atrtrib2=value2;attrib3=value3};
annotation.start = startNode;
annotation.end = endNode;
\end{verbatim}
\end{small}
\nnormalsize

The startNode and endNode are created from offsets referring the
beginning and the end of `A piece of text' in the document's
content.

The documents supported by GATE have to be in one of the encodings
accepted by Java. The most popular is the {\em `UTF-8'} encoding
which is also the most storage efficient one for UNICODE. If, when
loading a document in GATE the {\em encoding} parameter is set to
`'(the empty string), then the default encoding of the platform
will be used.


%%%%%%%%%%%%%%%%%%%%%%%%%%%%%%%%%%%%%%%%%%%%%%%%%%%%%%%%%%%%%%%%%%%%%%%%%%%%%
\subsect[sec:corpora:detecting-reader]{Detecting the Right Reader}
%%%%%%%%%%%%%%%%%%%%%%%%%%%%%%%%%%%%%%%%%%%%%%%%%%%%%%%%%%%%%%%%%%%%%%%%%%%%%

In order to successfully apply the document creation algorithm described above,
GATE needs to detect the proper reader to use for each document format.  If the
user knows in advance what kind of document they are loading then they can
specify the MIME type (e.g. {\em text/html}) using the init parameter
{\tt mimeType}, and GATE will respect this.  If an explicit type is not given,
GATE attempts to determine the type by other means, taking
%When opening a document in GATE, the file extension (e.g. {\tt xml})
%is important but if
%not present, GATE uses some other means to detect its type.
%In order to do that, it takes into
into consideration (where possible) the information provided by three
sources:
\begin{itemize}
\item
Document's extension
\item
The web server's content type
\item
Magic numbers detection
\end{itemize}

The first represents the extension of a file like ({\em
xml,htm,html,txt,sgm,rtf, etc}), the second represents the HTTP
information sent by a web server regarding the content type of the
document being send by it ({\em text/html; text/xml, etc}), and the
third one represents certain sequences of chars which are ultimately
number sequences. GATE is capable of supporting multimedia documents,
if the right reader is added to the framework.  Sometimes, multimedia
documents are identified by a signature consisting in a sequence of
numbers. Inside GATE they are called magic numbers. For textual
documents, certain char sequences form such magic numbers. Examples of
magic numbers sequences will be provided in the Input section of each
format supported by GATE.

All those tests are applied to each document read, and after that,
a voting mechanism decides what is the best reader to associate
with the document. There is a degree of priority for all those
tests. The document's extension test has the highest priority. If
the system is in doubt which reader to choose, then the one
associated with document's extension will be selected. The next
higher priority is given to the web server's content type and the
third one is given to the magic numbers detection. However, any
two tests that identify the same mime type, will have the highest
priority in deciding the reader that will be used. The web server
test is not always successful as there might be documents that are
loaded from a local file system, and the magic number detection
test is not always applicable. In the next paragraphs we will se
how those tests are performed and what is the general mechanism
behind reader detection.

The method that detects the proper reader is a static one, and it
belongs to the {\tt gate.DocumentFormat} class. It uses the
information stored in the maps filled by the init() method of each
reader. This method comes with three signatures:

\small
\begin{lstlisting}
static public DocumentFormat getDocumentFormat( gate.Document
aGateDocument, URL url)

static public DocumentFormat getDocumentFormat(gate.Document
aGateDocument, String fileSuffix)

static public DocumentFormat getDocumentFormat(gate.Document
aGateDocument, MimeType mimeType)

\end{lstlisting}
\nnormalsize

The first two methods try to detect the right MimeType for the
GATE document, and after that, they call the third one to return
the reader associate with a MimeType. Of course, if an explicit
{\tt mimeType} parameter was specified, GATE calls the third form
of the method directly, passing the specified type. GATE uses the
implementation from `http://jigsaw.w3.org' for mime types.

The magic numbers test is performed using the information form \\
magic2mimeTypeMap map. Each key from this map, is searched in the
first bufferSize (the default value is 2048) chars of text. The
method that does this is called \\ {\tt
runMagicNumbers(InputStreamReader aReader)} and it belongs to
DocumentFormat class. More details about it can be found in the
GATE API documentation.

In order to activate a reader to perform the unpacking, the creole
definition of a GATE document defines a parameter called
`markupAware' initialized with a default value of {\bf true}.
This parameter, forces GATE to detect a proper reader for the
document being read. If no reader is found, the document's content
is load and presented to the user, just like any other text editor
(this for textual documents).

You can also use Tika format auto-detection by setting the mimeType of a
document to "application/tika". Then the document will be parsed only by
Tika.

The next subsections investigates particularities for each format
and will describe the file extensions registered with each
document format.


%%%%%%%%%%%%%%%%%%%%%%%%%%%%%%%%%%%%%%%%%%%%%%%%%%%%%%%%%%%%%%%%%
\subsect[sec:corpora:xml]{XML}

%%%%%%%%%%%%%%%%%%%%%%%%%%%%%%%%%%%%%%%%%%%%%%%%%%%%%%%%%%%%%%%%%%%%%
\subsubsect[sec:corpora:input]{Input}

GATE permits the processing of any XML document and offers support
for XML namespaces. It benefits the power of Apache's Xerces
parser and also makes use of Sun's JAXP layer. Changing the XML
parser in GATE can be achieved by simply replacing the value of a
Java system property (`javax.xml.parsers.SAXParserFactory').

GATE will accept any well formed XML document as input. Although
it has the possibility to validate XML documents against DTDs it
does not do so because the validating procedure is time consuming
and in many cases it issues messages that are annoying for the
user.

There is an open problem with the general approach of reading XML, HTML and SGML
documents in GATE. As we previously said, the text covered by tags/elements is
appended to the GATE document content and a GATE annotation refers to this
particular span of text. When appending, in cases such as `{\tt
end.</P><P>Start}' it might happen that the ending word of the previous
annotation is concatenated with the beginning phrase of the annotation currently
being created, resulting in a garbage input for GATE processing resources that
operate at the text surface.

Let's take another example in order to better understand the
problem:

\small
\begin{small}
\begin{verbatim}
<title>This is a title</title><p>This is a paragraph</p><a
href="#link">Here is an useful link</a>
\end{verbatim}
\end{small}
\nnormalsize

When the markup is transformed to annotations, it is likely that
the text from the document's content will be as follows:

{\tt This is a titleThis is a paragraphHere is an useful link}

The annotations created will refer the right parts of the texts
but for the GATE's processing resources like (tokenizer, gazetteer,
etc) which work on this text, this will be a major disaster.
Therefore, in order to prevent this problem from happening, GATE
checks if it's likely to join words and if this happens then it
inserts a space between those words. So, the text will look like
this after loaded in GATE Developer:

{\tt This is a title This is a paragraph Here is an useful link}

There are cases when these words are meant to be joined, but they
are rare. This is why it's an open problem.

The extensions associate with the XML reader are:
\begin{itemize}
\item
xml
\item
xhtm
\item
xhtml
\end{itemize}

The web server content type associate with xml documents is: {\em
text/xml.}

The magic numbers test searches inside the document for the XML({\tt <?xml
version="1.0"}) signature. It is also able to detect if the XML document uses the
semantics described in the GATE document format DTD (see
\ref{sec:corpora:saveasxml} below) or uses other semantics.

\textbf{Namespace handling}

By default, GATE will retain the namespace prefix and namespace URIs of XML 
elements when creating annotations and features within the \textbf{Original markups} 
annotation set. For example, the element 
\begin{small}
\begin{verbatim}<dc:title xmlns:dc="http://purl.org/dc/elements/1.1/">Document title</dc:title>\end{verbatim}
\end{small}
will create the following annotation

\begin{small}
\begin{verbatim}dc:title(xmlns:dc=http://purl.org/dc/elements/1.1/)\end{verbatim}
\end{small}

However, as the colon character '{\tt :}' is a reserved meta-character in JAPE, it is
not possible to write a JAPE rule that will match the {\tt dc:title} element or its namespace URI.

If you need to match namespace-prefixed elements in the Original markups AS, you can
alter the default namespace deserialization behaviour to remove the namespace prefix and add it as a feature 
(along with the namespace URI), by specifying the following attributes in the {\tt<GATECONFIG>}
element of {\tt gate.xml} or local configuration file:


\begin{itemize}
\item \textbf{addNamespaceFeatures} - set to "true" to deserialize namespace prefix and uri information as features.
\item \textbf{namespaceURI} - The feature name to use that will hold the namespace URI of the element, e.g. "namespace"
\item \textbf{namespacePrefix} - The feature name to use that will hold the namespace prefix of the element, e.g. "prefix"
\end{itemize}
 
i.e.

\begin{small}
\begin{verbatim}<GATECONFIG 
	addNamespaceFeatures="true" 
	namespaceURI="namespace" 
	namespacePrefix="prefix" />\end{verbatim} 
\end{small}

For example

\begin{small}
\begin{verbatim}<dc:title>Document title</dc:title>\end{verbatim}
\end{small}
  
would create in Original markups AS (assuming the {\tt xmlns:dc} URI has defined in the document root or parent element)

\begin{small}
\begin{verbatim}title(prefix=dc, namespace=http://purl.org/dc/elements/1.1/)\end{verbatim} 
\end{small}

If a JAPE rule is written to create a new annotation, e.g.

\begin{small}
\begin{verbatim}description(prefix=foo, namespace=http://www.example.org/)\end{verbatim} 
\end{small}

then these would be serialized to

\begin{small}
\begin{verbatim}
<dc:title xmlns:dc="http://purl.org/dc/elements/1.1/">Document title</dc:title>
<foo:description xmlns:foo="http://www.example.org/">...</foo:description>
\end{verbatim}   
\end{small}

when using the '\textbf{Save preserving document format}' XML output option (see \ref{sec:corpora:saveasxml} below).  

  
%%%%%%%%%%%%%%%%%%%%%%%%%%%%%%%%%%%%%%%%%%%%%%%%%%%%%%%%%%%%%%%%%%%%%
\subsubsect[sec:corpora:output]{Output}

GATE is capable of ensuring persistence for its resources.
% These
%layers of persistence are various and they span until database
%persistence. However, for some purposes, a light and simple level of
%persistence would be highly appreciated. 
The types of persistent
storage used for Language Resources are:
%%%%%%%%%%%%%%%%%%
\begin{itemize}
%\item Databases (like Oracle);
\item Java serialization;
\item XML serialization.
\end{itemize}
%%%%%%%%%%%%%%%%%%
We describe the latter case here.

XML persistence doesn't necessarily preserve all the objects belonging to the
annotations, documents or corpora. Their features can be of all kinds of objects,
with various layers of nesting. For example, {\em lists containing lists
containing maps, etc}. Serializing these arbitrary data types in XML is not a
simple task; GATE does the best it can, and supports native Java types such as
Integers and Booleans, but where complex data types are used, information may be
lost(the types will be converted into Strings). GATE provides a full
serialization of certain types of features such as collections, strings and
numbers. It is possible to serialize only those collections containing strings or
numbers. The rest of other features are serialized using their string
representation and when read back, they will be all strings instead of being the
original objects. Consequences of this might be observed when performing
evaluations (see \Chapthing~\ref{chap:eval}).

When GATE outputs an XML document it may do so in one of two ways:
%
\begin{itemize}
%
\item
When the original document that was imported into GATE was an XML
document, GATE can dump that document back into XML (possibly with
additional markup added);
%
\item
For all document formats, GATE can dump its internal
representation of the document into XML.
\end{itemize}
%
In the former case, the XML output will be close to the original document. In
the latter case, the format is a GATE-specific one which can be read back by
the system to recreate all the information that GATE held internally for the
document.

In order to understand why there are two types of XML
serialization, one needs to understand the structure of a GATE
document. GATE allows a graph of annotations that refer to parts
of the text. Those annotations are grouped under annotation sets.
Because of this structure, sometimes it is impossible to save a
document as XML using tags that surround the text referred to by the
annotation, because tags crossover situations could appear (XML is
essentially a tree-based model of information, whereas GATE uses
graphs). Therefore, in order to preserve all annotations in a GATE
document, a custom type of XML document was developed.

The problem of crossover tags appears with GATE's second option
(the preserve format one), which is implemented at the cost of
losing certain annotations. The way it is applied in GATE is that
it tries to restore the original markup and where it is possible,
to add in the same manner annotations produced by GATE.

\paragraph{How to Access and Use the Two Forms of XML
Serialization}

\para[sec:corpora:saveasxml]{Save as XML Option}

This option is available in GATE Developer in the pop-up menu
associated with each language resource (document or corpus). Saving
a corpus as XML is done by calling `Save as XML' on each document of
the corpus. This option saves all the annotations of a document
together their features(applying the restrictions previously
discussed), using the GateDocument.dtd :

\small
\begin{small}
\begin{verbatim}
 <!ELEMENT GateDocument (GateDocumentFeatures,
           TextWithNodes, (AnnotationSet+))>
 <!ELEMENT GateDocumentFeatures (Feature+)>
 <!ELEMENT Feature (Name, Value)>
 <!ELEMENT Name (\#PCDATA)>
 <!ELEMENT Value (\#PCDATA)>
 <!ELEMENT TextWithNodes (\#PCDATA | Node)*>
 <!ELEMENT AnnotationSet (Annotation*)>
 <!ATTLIST AnnotationSet  Name CDATA \#IMPLIED>
 <!ELEMENT Annotation (Feature*)>
 <!ATTLIST Annotation  Type      CDATA \#REQUIRED
                       StartNode CDATA \#REQUIRED
                       EndNode   CDATA \#REQUIRED>
 <!ELEMENT Node EMPTY>
 <!ATTLIST Node id CDATA \#REQUIRED>
\end{verbatim}
\end{small}
\nnormalsize

The document is saved under a name chosen by the user and it may
have any extension. However, the recommended extension would be
`xml'.

Using GATE Embedded, this option is available by calling {\tt
gate.Document's toXml()} method. This method returns a string
which is the XML representation of the document on which the
method was called.

\noindent{\bf Note:} It is recommended that the string
representation to be saved on the file system using the UTF-8
encoding, as the first line of the string is : {\tt <?xml
version="1.0" encoding="UTF-8"?>}

{\em Example of such a GATE format document:}

\small
\begin{small}
\begin{verbatim}
<?xml version="1.0" encoding="UTF-8" ?>
<GateDocument>

<!-- The document's features-->

<GateDocumentFeatures>
<Feature>
  <Name className="java.lang.String">MimeType</Name>
  <Value className="java.lang.String">text/plain</Value>
</Feature>
<Feature>
  <Name className="java.lang.String">gate.SourceURL</Name>
  <Value className="java.lang.String">file:/G:/tmp/example.txt</Value>
</Feature>
</GateDocumentFeatures>

<!-- The document content area with serialized nodes -->

<TextWithNodes>
<Node id="0"/>A TEENAGER <Node
id="11"/>yesterday<Node id="20"/> accused his parents of cruelty
by feeding him a daily diet of chips which sent his weight
ballooning to 22st at the age of l2<Node id="146"/>.<Node
id="147"/>
</TextWithNodes>

<!-- The default annotation set -->

<AnnotationSet>
<Annotation Type="Date" StartNode="11"
EndNode="20">
<Feature>
  <Name className="java.lang.String">rule2</Name>
  <Value className="java.lang.String">DateOnlyFinal</Value>
</Feature> <Feature>
  <Name className="java.lang.String">rule1</Name>
  <Value className="java.lang.String">GazDateWords</Value>
</Feature> <Feature>
  <Name className="java.lang.String">kind</Name>
  <Value className="java.lang.String">date</Value>
</Feature> </Annotation> <Annotation Type="Sentence" StartNode="0"
EndNode="147"> </Annotation> <Annotation Type="Split"
StartNode="146" EndNode="147"> <Feature>
  <Name className="java.lang.String">kind</Name>
  <Value className="java.lang.String">internal</Value>
</Feature> </Annotation> <Annotation Type="Lookup" StartNode="11"
EndNode="20"> <Feature>
  <Name className="java.lang.String">majorType</Name>
  <Value className="java.lang.String">date_key</Value>
</Feature> </Annotation>
</AnnotationSet>

<!-- Named annotation set -->

<AnnotationSet Name="Original markups" >
 <Annotation
Type="paragraph" StartNode="0" EndNode="147"> </Annotation>
</AnnotationSet>
</GateDocument>
\end{verbatim}
\end{small}
\nnormalsize

\noindent{\bf Note:} One must know that all features that are not
collections containing numbers or strings or that are not numbers
or strings are discarded. With this option, GATE does not preserve
those features it cannot restore back.

\paragraph{The Preserve Format Option}

This option is available in GATE Developer from the popup menu of
the annotations table. If no annotation in this table is selected,
then the option will restore the document's original markup. If
certain annotations are selected, then the option will attempt to
restore the original markup and insert all the selected ones. When
an annotation violates the crossed over condition, that annotation
is discarded and a message is issued.

This option makes it possible to generate an XML document with tags
surrounding the annotation's referenced text and features saved as
attributes. All features which are collections, strings or numbers
are saved, and the others are discarded. However, when read back,
only the attributes under the GATE namespace (see below) are
reconstructed back differently to the others. That is because GATE
does not store in the XML document the information about the
features class and for collections the class of the items. So,
when read back, all features will become strings, except those
under the GATE namespace.

One will notice that all generated tags have an attribute called
`gateId' under the namespace `http://www.gate.ac.uk'. The
attribute is used when the document is read back in GATE, in order
to restore the annotation's old ID. This feature is needed because
it works in close cooperation with another attribute under the
same namespace, called `matches'. This attribute indicates
annotations/tags that refer the same entity\footnote{It's not an
XML entity but a information extraction named entity}. They are
under this namespace because GATE is sensitive to them and treats
them differently to all other elements with their attributes
which fall under the general reading algorithm described at the
beginning of this section.

The `gateId' under GATE namespace is used to create an
annotation which has as ID the value indicated by this
attribute. The `matches' attribute is used to create an
ArrayList in which the items will be Integers, representing the ID
of annotations that the current one matches.

{\em Example:}

If the text being processed is as follows:

\small
\begin{small}
\begin{verbatim}
<Person gate:gateId="23">John</Person> and <Person
gate:gateId="25" gate:matches="23;25;30">John Major</Person> are
the same person.
\end{verbatim}
\end{small}
\nnormalsize

What GATE does when it parses this text is it creates two
annotations:

\small
\begin{small}
\begin{verbatim}
a1.type = "Person"
a1.ID = Integer(23)
a1.start = <the start offset of
John>
a1.end = <the end offset of John>
a1.featureMap = {}

a2.type = "Person"
a2.ID = Integer(25)
a2.start = <the start offset
of John Major>
a2.end = <the end offset of John Major>
a2.featureMap = {matches=[Integer(23); Integer(25); Integer(30)]}

\end{verbatim}
\end{small}
\nnormalsize

Under GATE Embedded, this option is available by calling {\tt
gate.Document's toXml(Set aSetContainingAnnotations)} method. This
method returns a string which is the XML representation of the
document on which the method was called. If called with {\bf null}
as a parameter, then the method will attempt to restore only the
original markup. If the parameter is a set that contains
annotations, then each annotation is tested against the crossover
restriction, and for those found to violate it, a warning will be
issued and they will be discarded.

In the next subsections we will show how this option applies to
the other formats supported by GATE.

%%%%%%%%%%%%%%%%%%%%%%%%%%%%%%%%%%%%%%%%%%%%%%%%%%%%%%%%%%%%%%%%%%%%%
\subsect[sec:corpora:html]{HTML}
%%%%%%%%%%%%%%%%%%%%%%%%%%%%%%%%%%%%%%%%%%%%%%%%%%%%%%%%%%%%%%%%%%%%%

\subsubsect{Input}

HTML documents are parsed by GATE using the
\htlink{http://people.apache.org/~andyc/neko/doc/html/}{NekoHTML} parser.
%The parser used to access HTML documents is the one provided by
%Java.
The documents are read and created in GATE the same way as
the XML documents.

The extensions associate with the HTML reader are:
\begin{itemize}
\item
htm
\item
html
\end{itemize}

The web server content type associate with html documents is: {\em
text/html}.

The magic numbers test searches inside the document for the
HTML({\tt <html}) signature.There are certain HTML documents that
do not contain the HTML tag, so the magical numbers test might not
hold.

There is a certain degree of customization for HTML documents in
that GATE introduces new lines into the document's text content in
order to obtain a readable form. The annotations will refer the
pieces of text as described in the original document but there
will be a few extra new line characters inserted.

After reading H1, H2, H3, H4, H5, H6, TR, CENTER, LI, BR and DIV tags, GATE will
introduce a new line (NL) char into the text. After a TITLE tag it will
introduce two NLs. With P tags, GATE will introduce one NL at the beginning of the
paragraph and one at the end of the paragraph. All newly added NLs are not
considered to be part of the text contained by the tag.

\subsubsect{Output}

The `Save as XML' option works exactly the same for all GATE's
documents so there is no particular observation to be made for the
HTML formats.

When attempting to preserve the original markup formatting, GATE
will generate the document in xhtml. The html document will look
the same with any browser after processed by GATE but it will be
in another syntax.


%%%%%%%%%%%%%%%%%%%%%%%%%%%%%%%%%%%%%%%%%%%%%%%%%%%%%%%%%%%%%%%%%%%%%
\subsect[sec:corpora:sgml]{SGML}
%%%%%%%%%%%%%%%%%%%%%%%%%%%%%%%%%%%%%%%%%%%%%%%%%%%%%%%%%%%%%%%%%%%%%

\subsubsect{Input}

The SGML support in GATE is fairly light as there is no freely
available Java SGML parser. GATE uses a light converter attempting
to transform the input SGML file into a well formed XML. Because
it does not make use of a DTD, the conversion might not be always
good. It is advisable to perform a SGML2XML conversion outside the
system(using some other specialized tools) before using the SGML
document inside GATE.

The extensions associate with the SGML reader are:
\begin{itemize}
\item
sgm
\item
sgml
\end{itemize}

The web server content type associate with xml documents is : {\em
text/sgml}.

There is no  magic numbers test for SGML.

\subsubsect{Output}

When attempting to preserve the original markup formatting, GATE
will generate the document as XML because the real input of a SGML
document inside GATE is an XML one.


%%%%%%%%%%%%%%%%%%%%%%%%%%%%%%%%%%%%%%%%%%%%%%%%%%%%%%%%%%%%%%%%%%%%%
\subsect[sec:corpora:plain-text]{Plain text}
%%%%%%%%%%%%%%%%%%%%%%%%%%%%%%%%%%%%%%%%%%%%%%%%%%%%%%%%%%%%%%%%%%%%%

\subsubsect{Input}

When reading a plain text document, GATE attempts to detect its
paragraphs and add `paragraph' annotations to the document's
`Original markups' annotation set. It does that by detecting two
consecutive NLs. The procedure works for both UNIX like or DOS
like text files.

{\em Example:}

If the plain text read is as follows:

\small
\begin{small}
\begin{verbatim}
Paragraph 1. This text belongs to the first paragraph.

Paragraph 2. This text belongs to the second paragraph
\end{verbatim}
\end{small}
\nnormalsize

then two `paragraph' type annotation will be created in the
`Original markups' annotation set (referring the first and
second paragraphs ) with an empty feature map.

The extensions associate with the plain text reader are:
\begin{itemize}
\item
txt
\item
text
\end{itemize}

The web server content type associate with plain text documents
is: {\em text/plain.}

There is no magic numbers test for plain text.

\subsubsect{Output}

When attempting to preserve the original markup formatting, GATE
will dump XML markup that surrounds the text refereed.

The procedure described above applies both for plain text and RTF
documents.


%%%%%%%%%%%%%%%%%%%%%%%%%%%%%%%%%%%%%%%%%%%%%%%%%%%%%%%%%%%%%%%%%%%%%
\subsect[sec:corpora:rtf]{RTF}
%%%%%%%%%%%%%%%%%%%%%%%%%%%%%%%%%%%%%%%%%%%%%%%%%%%%%%%%%%%%%%%%%%%%%

\subsubsect{Input}

Accessing RTF documents is performed by using the Java's RTF
editor kit. It only extracts the document's text content from the
RTF document.

The extension associate with the RTF reader is {\em `rtf'}.

The web server content type associate with xml documents is : {\em
text/rtf}.

The magic numbers test searches for {\tt
\{$\backslash$$\backslash$rtf1}.

\subsubsect{Output}

Same as the plain tex output.


%%%%%%%%%%%%%%%%%%%%%%%%%%%%%%%%%%%%%%%%%%%%%%%%%%%%%%%%%%%%%%%%%%%%%
\subsect[sec:corpora:email]{Email}
%%%%%%%%%%%%%%%%%%%%%%%%%%%%%%%%%%%%%%%%%%%%%%%%%%%%%%%%%%%%%%%%%%%%%

\subsubsect{Input}

GATE is able to read email messages packed in one document (UNIX
mailbox format). It detects multiple messages inside such
documents and for each message it creates annotations for all the
fields composing an e-mail, like date, from, to, subject, etc. The
message's body is analyzed and a paragraph detection is performed
(just like in the plain text case) . All annotation created have
as type the name of the e-mail's fields and they are placed in the
Original markup annotation set.

{\em Example:}

\small
\begin{small}
\begin{verbatim}
From someone@zzz.zzz.zzz Wed Sep  6 10:35:50 2000

Date: Wed, 6 Sep2000 10:35:49 +0100 (BST)

From: forename1 surname2 <someone1@yyy.yyy.xxx>

To: forename2 surname2 <someone2@ddd.dddd.dd.dd>

Subject: A subject

Message-ID: <Pine.SOL.3.91.1000906103251.26010A-100000@servername>
MIME-Version: 1.0
Content-Type: TEXT/PLAIN; charset=US-ASCII

This text belongs to the e-mail body....

This is a paragraph in the body of the e-mail

This is another paragraph.
\end{verbatim}
\end{small}
\nnormalsize

GATE attempts to detect lines such as `{\em From someone@zzz.zzz.zzz Wed Sep  6
10:35:50 2000}' in the e-mail text. Those lines separate e-mail messages
contained in one file. After that, for each field in the e-mail message
annotations are created as follows:

The annotation type will be the name of the field, the feature map
will be empty and the annotation will span from the end of the
field until the end of the line containing the e-mail field.

{\em Example:}

\small
\begin{small}
\begin{verbatim}
a1.type = "date" a1 spans between the two ^ ^. Date:^ Wed,
6Sep2000 10:35:49 +0100 (BST)^

a2.type = "from"; a2 spans between the two ^ ^. From:^ forename1
surname2 <someone1@yyy.yyy.xxx>^
\end{verbatim}
\end{small}
\nnormalsize

The extensions associated with the email reader are:
\begin{itemize}
\item
eml
\item
email
\item
mail
\end{itemize}

The web server content type associate with plain text documents
is: {\em text/email.}

The magic numbers test searches for keywords like {\em
Subject:},etc.

\subsubsect{Output}

Same as plain text output.

%%%%%%%%%%%%%%%%%%%%%%%%%%%%%%%%%%%%%%%%%%%%%%%%%%%%%%%%%%%%%%%%%%%%%%%%%%%
\subsect[sec:corpora:tika]{PDF Files and Office Documents}
%%%%%%%%%%%%%%%%%%%%%%%%%%%%%%%%%%%%%%%%%%%%%%%%%%%%%%%%%%%%%%%%%%%%%%%%%%%

GATE uses the \htlink{http://tika.apache.org/}{Apache Tika} library to provide
support for PDF documents and a number of the document formats from both
Microsoft Office and OpenOffice. In essense Tika converts the document structure
into HTML which is then used to create a GATE document. This means that whilst
a PDF or Word document may have been loaded the ``Original markups'' set will
contain HTML elements. One advantage of this approach is that processing
resources and JAPE grammars designed for use with HTML files should also
work well with PDF and Office documents.

%%%%%%%%%%%%%%%%%%%%%%%%%%%%%%%%%%%%%%%%%%%%%%%%%%%%%%%%%%%%%%%%%%%%%%%%%%%
\subsect[sec:corpora:uima]{UIMA CAS Documents}
%%%%%%%%%%%%%%%%%%%%%%%%%%%%%%%%%%%%%%%%%%%%%%%%%%%%%%%%%%%%%%%%%%%%%%%%%%%
GATE can read UIMA CAS documents. The CAS stands for Common Analysis
Structure. It provides a common representation to the artifact being
analyzed, here a text.

The subject of analysis (SOFA), here a string, is used as the document
content. Multiple sofa are concatenated. The analysis results or metadata
are added as annotations when having begin and end offsets and otherwise are
added as document features. The views are added as GATE annotation sets.
The type system (a hierarchical annotation schema) is not currently
supported.

The web server content type associate with UIMA documents
is: {\em text/xmi+xml.}

The extensions are: xcas, xmicas, xmi.

The magic numbers are:
\begin{verbatim}
<CAS version="2">
\end{verbatim}
and
\begin{verbatim}
xmlns:cas=
\end{verbatim}

%%%%%%%%%%%%%%%%%%%%%%%%%%%%%%%%%%%%%%%%%%%%%%%%%%%%%%%%%%%%%%%%%%%%%%%%%%%
\subsect[sec:corpora:conll]{CoNLL/IOB Documents}
%%%%%%%%%%%%%%%%%%%%%%%%%%%%%%%%%%%%%%%%%%%%%%%%%%%%%%%%%%%%%%%%%%%%%%%%%%%
GATE can read files of text annotated in the traditional CoNLL or BIO/BILOU
format, typically used to represent POS tags and chunks and best known for
Conference on Natural Language
Learning\footnote{\url{http://ifarm.nl/signll/conll/}} tasks.  The following
example illustrates one sentence with POS and chunk tags (\texttt{B-} and
\texttt{I-} indicate the beginning and continuation, respectively, of a chunk);
the columns represent the tokens, the POS tags, and the chunk tags, and
sentences are separated by blank lines.
%%
\begin{small}
\begin{verbatim}
My    PRP$  B-NP
dog   NN    I-NP
has   VBZ   B-VP
fleas NNS   B-NP
.     .     O
\end{verbatim}
\end{small}


GATE interpets this format quite flexibly: the columns can be separated by any
whitespace sequence, and the number of columns can vary.  The strings from the
leftmost column become strings in the document content, with spaces interposed,
and Token and SpaceToken annotations (with \emph{string} and \emph{length}
features) are created appropriately in the \emph{Original markups} set).  

Each blank line (empty or containing only whitespace) in the original data
becomes a newline in the document content.

The tags in subsequent columns are transformed into annotations.  A chunk tag
(beginning with \texttt{B-} and followed by zero or more matching \texttt{I-}
tags) produces an annotation whose type is determined by the rest of the tag
(\texttt{NP} or \texttt{VP} in the above example, but any string with no
whitespace is acceptable), with a \emph{kind = chunk} feature.  A chunk tag
beginning with \texttt{L-} (\emph{last}) terminates the chunk, and a \texttt{U-}
(\emph{unigram}) tag produces a chunk annotation over one token.  Other tags
produce annotations with the tag name as the type and a \emph{kind = token}
feature.

Every annotation derived from a tag has a \emph{column} feature whose
\texttt{int} value indicates the source column in the data (numbered from 0 for
the string column).  An ``\texttt{O}'' tag closes all open chunk tags at the end
of the previous token.

This document format is associated with MIME-type \texttt{text/x-conll} and
filename extensions \texttt{.conll} and \texttt{.iob}.
%%
%%%%%%%%%%%%%%%%%%%%%%%%%%%%%%%%%%%%%%%%%%%%%%%%%%%%%%%%%%%%%%%%%%%%%%%%%%%%%
\sect[sec:corpora:xmlinout]{XML Input/Output}
%%%%%%%%%%%%%%%%%%%%%%%%%%%%%%%%%%%%%%%%%%%%%%%%%%%%%%%%%%%%%%%%%%%%%%%%%%%%%

Support for input from and output to XML is described in Section
\ref{sec:corpora:xml}. In short: 
%
\begin{itemize}
%
\item
GATE will read any well-formed XML document (it does not attempt to validate
XML documents). Markup will by default be converted into native GATE format.
%
\item
GATE will write back into XML in one of two ways:
  \begin{enumerate}
  \item
  Preserving the original format and adding selected markup (for example to
  add the results of some language analysis process to the document).
  \item
  In GATE's own XML serialisation format, which encodes all the data in a
  GATE Document (as far as this is possible within a tree-structured
  paradigm -- for 100\% non-lossy data storage use GATE's RDBMS or binary
  serialisation facilities -- see Section~\ref{sec:creole-model:datastores}).
  \end{enumerate}
%
\end{itemize}
%
When using GATE Embedded, object representations of XML documents such
as {\tt DOM} or {\tt jDOM}, or query and transformation languages such
as {\tt X-Path} or {\tt XSLT}, may be used in parallel with GATE's own
Document representation ({\tt gate.Document}) without conflicts.

