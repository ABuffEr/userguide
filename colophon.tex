
%%%%%%%%%%%%%%%%%%%%%%%%%%%%%%%%%%%%%%%%%%%%%%%%%%%%%%%%%%%%%%%%%%%%%%%%%%%%%
%
% colophon.tex
%
% hamish, 25/8/1
%
% $Id: colophon.tex,v 1.3 2002/03/06 12:06:36 hamish Exp $
%
%%%%%%%%%%%%%%%%%%%%%%%%%%%%%%%%%%%%%%%%%%%%%%%%%%%%%%%%%%%%%%%%%%%%%%%%%%%%%


%%%%%%%%%%%%%%%%%%%%%%%%%%%%%%%%%%%%%%%%%%%%%%%%%%%%%%%%%%%%%%%%%%%%%%%%%%%%%
%\addcontentsline{toc}{chapter}{Colophon}
\chapter*{Colophon}
\markboth{}{}
%%%%%%%%%%%%%%%%%%%%%%%%%%%%%%%%%%%%%%%%%%%%%%%%%%%%%%%%%%%%%%%%%%%%%%%%%%%%%

%%%% qqqqqqqqqqqqqqqqqqqqqqqqq %%%%
\begin{quote}
Formal semantics (henceforth FS), at least as it relates to computational
language understanding, is in one way rather like
connectionism, though without the crucial prop Sejnowski's work (1986)
is widely believed to give to the latter: both are old
doctrines returned, like the Bourbons, having learned nothing and forgotten
nothing. But FS has nothing to show as a showpiece of
success after all the intellectual groaning and effort. 

{\it On Keeping Logic in its Place} (in Theoretical Issues in Natural Language
Processing, ed. Wilks), Yorick Wilks, 1989 (p.130).
\end{quote}
%%%% qqqqqqqqqqqqqqqqqqqqqqqqq %%%%


We used \LaTeX\ to produce this document, along with
\htlink{http://www.cse.ohio-state.edu/~gurari/TeX4ht/}{TeX4HT}
for the HTML production. Thank you Don Knuth, Leslie Lamport and Eitan Gurari.

