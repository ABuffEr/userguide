%%%%%%%%%%%%%%%%%%%%%%%%%%%%%%%%%%%%%%%%%%%%%%%%%%%%%%%%%%%%%%%%%%%%%%%%%%%%%
%
% mlconfig.tex
%
% diana, December 2002
%
% $I $
%
%%%%%%%%%%%%%%%%%%%%%%%%%%%%%%%%%%%%%%%%%%%%%%%%%%%%%%%%%%%%%%%%%%%%%%%%%%%%%


%%%%%%%%%%%%%%%%%%%%%%%%%%%%%%%%%%%%%%%%%%%%%%%%%%%%%%%%%%%%%%%%%%%%%%%%%%%%%
\chapt[chap:mlconfig]{Sample ML Configuration File}
\markboth{Sample ML Configuration File}{Sample ML Configuration File}
%%%%%%%%%%%%%%%%%%%%%%%%%%%%%%%%%%%%%%%%%%%%%%%%%%%%%%%%%%%%%%%%%%%%%%%%%%%%%

This example config file is for use with the Machine Learning PR in the
`Machine\_Learning' plugin. GATE also provides a new machine learning PR,
called Batch Learning PR. This can be found in the `Learning' plugin, and example
configuration files are provided in \Chapthing~\ref{chap:ml}, in the form of
three case studies, see Section \ref{sec:ml:batch-case-studies}. 

\begin{verbatim}
<?xml version="1.0" encoding="UTF-8"?>
<ML-CONFIG>
  <DATASET>
  <!-- The type of annotation used as instance -->
  <INSTANCE-TYPE>Token</INSTANCE-TYPE>
  <ATTRIBUTE>
    <!-- The name given to the attribute -->
    <NAME>Lookup(0)</NAME>
    <!-- The type of annotation used as attribute -->
    <TYPE>Lookup</TYPE>
    <!-- The position relative to the instance annotation -->
    <POSITION>0</POSITION>
  </ATTRIBUTE>


  <ATTRIBUTE>
    <!-- The name given to the attribute -->
    <NAME>Lookup_MT(-1)</NAME>
    <!-- The type of annotation used as attribute -->
    <TYPE>Lookup</TYPE>
    <!-- Optional: the feature name for the feature used to extract values
    for the attribute -->
    <FEATURE>majorType</FEATURE>

    <!-- The position relative to the instance annotation -->
    <POSITION>-1</POSITION>
    <!-- The list of permitted values.
    if present, marks a nominal attribute;
    if absent, the attribute is numeric (double)        -->
    <VALUES>
      <!-- One permitted value -->
      <VALUE>address</VALUE>
      <VALUE>cdg</VALUE>
      <VALUE>country_adj</VALUE>
      <VALUE>currency_unit</VALUE>
      <VALUE>date</VALUE>
      <VALUE>date_key</VALUE>
      <VALUE>date_unit</VALUE>
      <VALUE>facility</VALUE>
      <VALUE>facility_key</VALUE>
      <VALUE>facility_key_ext</VALUE>
      <VALUE>govern_key</VALUE>
      <VALUE>greeting</VALUE>
      <VALUE>ident_key</VALUE>
      <VALUE>jobtitle</VALUE>
      <VALUE>loc_general_key</VALUE>
      <VALUE>loc_key</VALUE>
      <VALUE>location</VALUE>
      <VALUE>number</VALUE>
      <VALUE>org_base</VALUE>
      <VALUE>org_ending</VALUE>
      <VALUE>org_key</VALUE>
      <VALUE>org_pre</VALUE>
      <VALUE>organization</VALUE>
      <VALUE>organization_noun</VALUE>
      <VALUE>person_ending</VALUE>
      <VALUE>person_first</VALUE>
      <VALUE>person_full</VALUE>
      <VALUE>phone_prefix</VALUE>
      <VALUE>sport</VALUE>
      <VALUE>spur</VALUE>
      <VALUE>spur_ident</VALUE>
      <VALUE>stop</VALUE>
      <VALUE>surname</VALUE>
      <VALUE>time</VALUE>
      <VALUE>time_modifier</VALUE>
      <VALUE>time_unit</VALUE>
      <VALUE>title</VALUE>
      <VALUE>year</VALUE>
    </VALUES>
    <!-- Optional: if present marks the attribute used as CLASS
    Only one attribute can be marked as class -->
  </ATTRIBUTE>

  <ATTRIBUTE>
    <!-- The name given to the attribute -->
    <NAME>Lookup_MT(0)</NAME>
    <!-- The type of annotation used as attribute -->
    <TYPE>Lookup</TYPE>
    <!-- Optional: the feature name for the feature used to extract values
    for the attribute -->
    <FEATURE>majorType</FEATURE>

    <!-- The position relative to the instance annotation -->
    <POSITION>0</POSITION>
    <!-- The list of permitted values.
    if present, marks a nominal attribute;
    if absent, the attribute is numeric (double)        -->
    <VALUES>
      <!-- One permitted value -->
          <VALUE>address</VALUE>
      <VALUE>cdg</VALUE>
      <VALUE>country_adj</VALUE>
      <VALUE>currency_unit</VALUE>
      <VALUE>date</VALUE>
      <VALUE>date_key</VALUE>
      <VALUE>date_unit</VALUE>
      <VALUE>facility</VALUE>
      <VALUE>facility_key</VALUE>
      <VALUE>facility_key_ext</VALUE>
      <VALUE>govern_key</VALUE>
      <VALUE>greeting</VALUE>
      <VALUE>ident_key</VALUE>
      <VALUE>jobtitle</VALUE>
      <VALUE>loc_general_key</VALUE>
      <VALUE>loc_key</VALUE>
      <VALUE>location</VALUE>
      <VALUE>number</VALUE>
      <VALUE>org_base</VALUE>
      <VALUE>org_ending</VALUE>
      <VALUE>org_key</VALUE>
      <VALUE>org_pre</VALUE>
      <VALUE>organization</VALUE>
      <VALUE>organization_noun</VALUE>
      <VALUE>person_ending</VALUE>
      <VALUE>person_first</VALUE>
      <VALUE>person_full</VALUE>
      <VALUE>phone_prefix</VALUE>
      <VALUE>sport</VALUE>
      <VALUE>spur</VALUE>
      <VALUE>spur_ident</VALUE>
      <VALUE>stop</VALUE>
      <VALUE>surname</VALUE>
      <VALUE>time</VALUE>
      <VALUE>time_modifier</VALUE>
      <VALUE>time_unit</VALUE>
      <VALUE>title</VALUE>
      <VALUE>year</VALUE>
    </VALUES>
    <!-- Optional: if present marks the attribute used as CLASS
    Only one attribute can be marked as class -->
  </ATTRIBUTE>

  <ATTRIBUTE>
    <!-- The name given to the attribute -->
    <NAME>Lookup_MT(1)</NAME>
    <!-- The type of annotation used as attribute -->
    <TYPE>Lookup</TYPE>
    <!-- Optional: the feature name for the feature used to extract values
    for the attribute -->
    <FEATURE>majorType</FEATURE>

    <!-- The position relative to the instance annotation -->
    <POSITION>1</POSITION>

    <!-- The list of permitted values.
    if present, marks a nominal attribute;
    if absent, the attribute is numeric (double)        -->
    <VALUES>
      <!-- One permitted value -->
      <VALUE>address</VALUE>
      <VALUE>cdg</VALUE>
      <VALUE>country_adj</VALUE>
      <VALUE>currency_unit</VALUE>
      <VALUE>date</VALUE>
      <VALUE>date_key</VALUE>
      <VALUE>date_unit</VALUE>
      <VALUE>facility</VALUE>
      <VALUE>facility_key</VALUE>
      <VALUE>facility_key_ext</VALUE>
      <VALUE>govern_key</VALUE>
      <VALUE>greeting</VALUE>
      <VALUE>ident_key</VALUE>
      <VALUE>jobtitle</VALUE>
      <VALUE>loc_general_key</VALUE>
      <VALUE>loc_key</VALUE>
      <VALUE>location</VALUE>
      <VALUE>number</VALUE>
      <VALUE>org_base</VALUE>
      <VALUE>org_ending</VALUE>
      <VALUE>org_key</VALUE>
      <VALUE>org_pre</VALUE>
      <VALUE>organization</VALUE>
      <VALUE>organization_noun</VALUE>
      <VALUE>person_ending</VALUE>
      <VALUE>person_first</VALUE>
      <VALUE>person_full</VALUE>
      <VALUE>phone_prefix</VALUE>
      <VALUE>sport</VALUE>
      <VALUE>spur</VALUE>
      <VALUE>spur_ident</VALUE>
      <VALUE>stop</VALUE>
      <VALUE>surname</VALUE>
      <VALUE>time</VALUE>
      <VALUE>time_modifier</VALUE>
      <VALUE>time_unit</VALUE>
      <VALUE>title</VALUE>
      <VALUE>year</VALUE>
    </VALUES>
    <!-- Optional: if present marks the attribute used as CLASS
    Only one attribute can be marked as class -->
  </ATTRIBUTE>

  <ATTRIBUTE>
    <!-- The name given to the attribute -->
    <NAME>POS_category(-1)</NAME>
    <!-- The type of annotation used as attribute -->
    <TYPE>Token</TYPE>
    <!-- Optional: the feature name for the feature used to extract values
    for the attribute -->
    <FEATURE>category</FEATURE>

    <!-- The position relative to the instance annotation -->
    <POSITION>-1</POSITION>

    <!-- The list of permitted values.
    if present, marks a nominal attribute;
    if absent, the attribute is numeric (double)        -->
    <VALUES>
      <!-- One permitted value -->
        <VALUE>NN</VALUE>
        <VALUE>NNP</VALUE>
        <VALUE>NNPS</VALUE>
        <VALUE>NNS</VALUE>
        <VALUE>NP</VALUE>
        <VALUE>NPS</VALUE>
        <VALUE>JJ</VALUE>
        <VALUE>JJR</VALUE>
        <VALUE>JJS</VALUE>
        <VALUE>JJSS</VALUE>
        <VALUE>RB</VALUE>
        <VALUE>RBR</VALUE>
        <VALUE>RBS</VALUE>
        <VALUE>VB</VALUE>
        <VALUE>VBD</VALUE>
        <VALUE>VBG</VALUE>
        <VALUE>VBN</VALUE>
        <VALUE>VBP</VALUE>
        <VALUE>VBZ</VALUE>
        <VALUE>FW</VALUE>
        <VALUE>CD</VALUE>
        <VALUE>CC</VALUE>
        <VALUE>DT</VALUE>
        <VALUE>EX</VALUE>
        <VALUE>IN</VALUE>
        <VALUE>LS</VALUE>
        <VALUE>MD</VALUE>
        <VALUE>PDT</VALUE>
        <VALUE>POS</VALUE>
        <VALUE>PP</VALUE>
        <VALUE>PRP</VALUE>
        <VALUE>PRP$</VALUE>
        <VALUE>PRPR$</VALUE>
        <VALUE>RP</VALUE>
        <VALUE>TO</VALUE>
        <VALUE>UH</VALUE>
        <VALUE>WDT</VALUE>
        <VALUE>WP</VALUE>
        <VALUE>WP$</VALUE>
        <VALUE>WRB</VALUE>
        <VALUE>SYM</VALUE>
        <VALUE>\"</VALUE>
        <VALUE>#</VALUE>
        <VALUE>$</VALUE>
        <VALUE>'</VALUE>
        <VALUE>(</VALUE>
        <VALUE>)</VALUE>
        <VALUE>,</VALUE>
        <VALUE>--</VALUE>
        <VALUE>-LRB-</VALUE>
        <VALUE>.</VALUE>
        <VALUE>'</VALUE>
        <VALUE>:</VALUE>
        <VALUE>::</VALUE>
        <VALUE>`</VALUE>
    </VALUES>
    <!-- Optional: if present marks the attribute used as CLASS
    Only one attribute can be marked as class -->
  </ATTRIBUTE>

  <ATTRIBUTE>
    <!-- The name given to the attribute -->
    <NAME>POS_category(0)</NAME>
    <!-- The type of annotation used as attribute -->
    <TYPE>Token</TYPE>
    <!-- Optional: the feature name for the feature used to extract values
    for the attribute -->
    <FEATURE>category</FEATURE>

    <!-- The position relative to the instance annotation -->
    <POSITION>0</POSITION>

    <!-- The list of permitted values.
    if present, marks a nominal attribute;
    if absent, the attribute is numeric (double)        -->
    <VALUES>
      <!-- One permitted value -->
        <VALUE>NN</VALUE>
        <VALUE>NNP</VALUE>
        <VALUE>NNPS</VALUE>
        <VALUE>NNS</VALUE>
        <VALUE>NP</VALUE>
        <VALUE>NPS</VALUE>
        <VALUE>JJ</VALUE>
        <VALUE>JJR</VALUE>
        <VALUE>JJS</VALUE>
        <VALUE>JJSS</VALUE>
        <VALUE>RB</VALUE>
        <VALUE>RBR</VALUE>
        <VALUE>RBS</VALUE>
        <VALUE>VB</VALUE>
        <VALUE>VBD</VALUE>
        <VALUE>VBG</VALUE>
        <VALUE>VBN</VALUE>
        <VALUE>VBP</VALUE>
        <VALUE>VBZ</VALUE>
        <VALUE>FW</VALUE>
        <VALUE>CD</VALUE>
        <VALUE>CC</VALUE>
        <VALUE>DT</VALUE>
        <VALUE>EX</VALUE>
        <VALUE>IN</VALUE>
        <VALUE>LS</VALUE>
        <VALUE>MD</VALUE>
        <VALUE>PDT</VALUE>
        <VALUE>POS</VALUE>
        <VALUE>PP</VALUE>
        <VALUE>PRP</VALUE>
        <VALUE>PRP$</VALUE>
        <VALUE>PRPR$</VALUE>
        <VALUE>RP</VALUE>
        <VALUE>TO</VALUE>
        <VALUE>UH</VALUE>
        <VALUE>WDT</VALUE>
        <VALUE>WP</VALUE>
        <VALUE>WP$</VALUE>
        <VALUE>WRB</VALUE>
        <VALUE>SYM</VALUE>
        <VALUE>\"</VALUE>
        <VALUE>#</VALUE>
        <VALUE>$</VALUE>
        <VALUE>'</VALUE>
        <VALUE>(</VALUE>
        <VALUE>)</VALUE>
        <VALUE>,</VALUE>
        <VALUE>--</VALUE>
        <VALUE>-LRB-</VALUE>
        <VALUE>.</VALUE>
        <VALUE>'</VALUE>
        <VALUE>:</VALUE>
        <VALUE>::</VALUE>
        <VALUE>`</VALUE>
    </VALUES>
    <!-- Optional: if present marks the attribute used as CLASS
    Only one attribute can be marked as class -->
  </ATTRIBUTE>

  <ATTRIBUTE>
    <!-- The name given to the attribute -->
    <NAME>POS_category(1)</NAME>
    <!-- The type of annotation used as attribute -->
    <TYPE>Token</TYPE>
    <!-- Optional: the feature name for the feature used to extract values
    for the attribute -->
    <FEATURE>category</FEATURE>

    <!-- The position relative to the instance annotation -->
    <POSITION>1</POSITION>

    <!-- The list of permitted values.
    if present, marks a nominal attribute;
    if absent, the attribute is numeric (double)        -->
    <VALUES>
      <!-- One permitted value -->
        <VALUE>NN</VALUE>
        <VALUE>NNP</VALUE>
        <VALUE>NNPS</VALUE>
        <VALUE>NNS</VALUE>
        <VALUE>NP</VALUE>
        <VALUE>NPS</VALUE>
        <VALUE>JJ</VALUE>
        <VALUE>JJR</VALUE>
        <VALUE>JJS</VALUE>
        <VALUE>JJSS</VALUE>
        <VALUE>RB</VALUE>
        <VALUE>RBR</VALUE>
        <VALUE>RBS</VALUE>
        <VALUE>VB</VALUE>
        <VALUE>VBD</VALUE>
        <VALUE>VBG</VALUE>
        <VALUE>VBN</VALUE>
        <VALUE>VBP</VALUE>
        <VALUE>VBZ</VALUE>
        <VALUE>FW</VALUE>
        <VALUE>CD</VALUE>
        <VALUE>CC</VALUE>
        <VALUE>DT</VALUE>
        <VALUE>EX</VALUE>
        <VALUE>IN</VALUE>
        <VALUE>LS</VALUE>
        <VALUE>MD</VALUE>
        <VALUE>PDT</VALUE>
        <VALUE>POS</VALUE>
        <VALUE>PP</VALUE>
        <VALUE>PRP</VALUE>
        <VALUE>PRP$</VALUE>
        <VALUE>PRPR$</VALUE>
        <VALUE>RP</VALUE>
        <VALUE>TO</VALUE>
        <VALUE>UH</VALUE>
        <VALUE>WDT</VALUE>
        <VALUE>WP</VALUE>
        <VALUE>WP$</VALUE>
        <VALUE>WRB</VALUE>
        <VALUE>SYM</VALUE>
        <VALUE>\"</VALUE>
        <VALUE>#</VALUE>
        <VALUE>$</VALUE>
        <VALUE>'</VALUE>
        <VALUE>(</VALUE>
        <VALUE>)</VALUE>
        <VALUE>,</VALUE>
        <VALUE>--</VALUE>
        <VALUE>-LRB-</VALUE>
        <VALUE>.</VALUE>
        <VALUE>'</VALUE>
        <VALUE>:</VALUE>
        <VALUE>::</VALUE>
        <VALUE>`</VALUE>
    </VALUES>
    <!-- Optional: if present marks the attribute used as CLASS
    Only one attribute can be marked as class -->
  </ATTRIBUTE>

  <ATTRIBUTE>
    <!-- The name given to the attribute -->
    <NAME>Entity(0)</NAME>
    <!-- The type of annotation used as attribute -->
    <TYPE>Entity</TYPE>
    <!-- The position relative to the instance annotation -->
    <POSITION>0</POSITION>

    <CLASS/>
    <!-- Optional: if present marks the attribute used as CLASS
    Only one attribute can be marked as class -->
  </ATTRIBUTE>


  </DATASET>

  <ENGINE>
    <WRAPPER>gate.creole.ml.weka.Wrapper</WRAPPER>
    <OPTIONS>
        <CLASSIFIER OPTIONS="-S -C 0.25 -B -M 2">weka.classifiers.trees.J48</CLASSIFIER>
        <CONFIDENCE-THRESHOLD>0.85</CONFIDENCE-THRESHOLD>
    </OPTIONS>
  </ENGINE>
</ML-CONFIG>
\end{verbatim}








