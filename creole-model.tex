%%%%%%%%%%%%%%%%%%%%%%%%%%%%%%%%%%%%%%%%%%%%%%%%%%%%%%%%%%%%%%%%%%%%%%%%%%%%%
%
% creole-model.tex
%
% hamish, 25/8/1
%
% $Id: creole-model.tex,v 1.16 2005/02/10 17:08:59 ian Exp $
%
%%%%%%%%%%%%%%%%%%%%%%%%%%%%%%%%%%%%%%%%%%%%%%%%%%%%%%%%%%%%%%%%%%%%%%%%%%%%%


%%%%%%%%%%%%%%%%%%%%%%%%%%%%%%%%%%%%%%%%%%%%%%%%%%%%%%%%%%%%%%%%%%%%%%%%%%%%%
\chapt[chap:creole-model]{CREOLE: the GATE Component Model}
\markboth{CREOLE: the GATE Component Model}{CREOLE: the GATE Component Model}
%%%%%%%%%%%%%%%%%%%%%%%%%%%%%%%%%%%%%%%%%%%%%%%%%%%%%%%%%%%%%%%%%%%%%%%%%%%%%

%%%% qqqqqqqqqqqqqqqqqqqqqqqqq %%%%
\ifprintedbook
\else
\begin{quote}
\ldots Noam Chomsky's answer in {\em Secrets, Lies and Democracy}
(David Barsamian 1994; Odonian) to `What do you think about the Internet?'

`I think that there are good things about it, but there are also
aspects of it that concern and worry me. This is an intuitive
response -- I can't prove it -- but my feeling is that, since people
aren't Martians or robots, direct face-to-face contact is an
extremely important part of human life. It helps develop self-understanding
and the growth of a healthy personality.

`You just have a different relationship to somebody when
you're looking at them than you do when you're punching away at a
keyboard and some symbols come back. I suspect that
extending that form of abstract and remote relationship, instead of
direct, personal contact, is going to have unpleasant
effects on what people are like. It will diminish their humanity, I think.'

Chomsky, quoted at \htlinkplain{http://philip.greenspun.com/wtr/dead-trees/53015}.
\end{quote}
\fi
%%%% qqqqqqqqqqqqqqqqqqqqqqqqq %%%%

The GATE architecture is based on components: reusable chunks of software
with well-defined interfaces that may be deployed in a variety of contexts.
The design of GATE is based on an analysis of previous work on
infrastructure for LE, and of the typical types of software entities
found in the fields of NLP and CL (see in particular chapters 4--6
of \cite{Cun00a}). Our research suggested that a profitable way
to support LE software development was an architecture that
breaks down such programs into components of various types.
Because LE practice varies very widely (it is, after all, predominantly
a research field), the architecture must avoid restricting the sorts
of components that developers can plug into the infrastructure.
The GATE framework accomplishes this via an adapted
version of the {\em Java Beans} component framework from Sun, as described in
section~\ref{sec:creole-model:framework}.

GATE components may be implemented by a variety of programming languages and
databases, but in each case they are represented to the system as a Java
class. This class may do nothing other than call the underlying program, or
provide an access layer to a database; on the other hand
it may implement the whole component.

GATE components are one of three types:
%
\begin{itemize}
%
\item
LanguageResources (LRs) represent entities such as lexicons, corpora or
ontologies;
%
\item
ProcessingResources (PRs) represent entities that are primarily algorithmic,
such as parsers, generators or ngram modellers;
%
\item
VisualResources (VRs) represent visualisation and editing components that
participate in GUIs.
%
\end{itemize}
%
\ifprintedbook
\else
The distinction between language resources and processing resources is explored more fully in section \ref{sec:design:components}.
\fi
Collectively, the set of resources integrated with GATE is known as
{\bf CREOLE}: a
Collection of REusable Objects for Language Engineering.

In the rest of this \chapthing:
\begin{itemize}
\item
Section \ref{sec:creole-model:lifecycle} describes the lifecycle of GATE
components;
\item
Section \ref{sec:creole-model:applications} describes how Processing Resources
can be grouped into applications;
\item
Section \ref{sec:creole-model:datastores} describes the relationship between
Language Resources and their datastores;
\item
Section \ref{sec:creole-model:builtins} summarises GATE's set of built-in
components;
\item
Section \ref{sec:creole-model:config} describes how configuration data for
Resource types is supplied to GATE.
\end{itemize}


%%%%%%%%%%%%%%%%%%%%%%%%%%%%%%%%%%%%%%%%%%%%%%%%%%%%%%%%%%%%%%%%%%%%%%%%%%%%%
\sect[sec:creole-model:creoleweb]{The Web and CREOLE}
%%%%%%%%%%%%%%%%%%%%%%%%%%%%%%%%%%%%%%%%%%%%%%%%%%%%%%%%%%%%%%%%%%%%%%%%%%%%%

GATE allows resource implementations and Language Resource persistent data to
be distributed over the Web, and uses Java annotations and XML for
configuration of resources (and GATE itself).

Resource implementations are grouped together as `plugins', stored
at a URL (when the resources are in the local file system this can be
a {\tt file:/} URL). When a plugin is loaded into GATE it looks for a
configuration file called {\tt creole.xml} relative to the plugin URL
and uses the contents of this file to determine what resources this
plugin declares and where to find the classes that implement the
resource types (typically these classes are stored in a JAR file in
the plugin directory).  Configuration data for the resources may be
stored directly in the creole.xml file, or it may be stored as Java
annotations on the resource classes themselves; in either case GATE
retrieves this configuration information and adds the resource
definitions to the CREOLE register. When a user requests an
instantiation of a resource, GATE creates an instance of the resource
class in the virtual machine.

Language resource data can be stored in binary serialised form in the local
file system.
%, or in an RDBMS like Oracle. In the latter case, communication
%with the database is over JDBC\footnote{The Java DataBase Connectivity
%layer.}, allowing the data to be located anywhere on the network (or anywhere
%you can get Oracle running, that is!).


%%%%%%%%%%%%%%%%%%%%%%%%%%%%%%%%%%%%%%%%%%%%%%%%%%%%%%%%%%%%%%%%%%%%%%%%%%%%%
\sect[sec:creole-model:framework]{The GATE Framework}
\label{sec:creole-model:beans}
%%%%%%%%%%%%%%%%%%%%%%%%%%%%%%%%%%%%%%%%%%%%%%%%%%%%%%%%%%%%%%%%%%%%%%%%%%%%%

We can think of the GATE framework as a backplane into
which users can plug CREOLE components.
The user gives the system a list of URLs to search when it starts up,
and components at those locations are loaded by the system.

The backplane performs these functions:
\begin{itemize}
\item
component discovery, bootstrapping, loading and reloading;
\item
management and visualisation of
native data structures for common information types;
\item
generalised data storage and process execution.
\end{itemize}

A set of components plus the framework is a deployment unit which can be
embedded in another application.

At their most basic, all GATE resources are {\em Java Beans}, the Java
platform's model of software components. Beans are simply Java classes that
obey certain interface conventions:
\begin{itemize}
\item
beans must have no-argument constructors.
%
\item
beans have \emph{properties}, defined by pairs of methods named by the
convention \texttt{set\emph{Prop}} and \texttt{get\emph{Prop}}.
\end{itemize}
%

GATE uses Java Beans conventions to construct and configure resources at
runtime, and defines interfaces that different component types must implement.

%%%%%%%%%%%%%%%%%%%%%%%%%%%%%%%%%%%%%%%%%%%%%%%%%%%%%%%%%%%%%%%%%%%%%%%%%%%%%
\sect[sec:creole-model:lifecycle]{The Lifecycle of a CREOLE Resource}
%%%%%%%%%%%%%%%%%%%%%%%%%%%%%%%%%%%%%%%%%%%%%%%%%%%%%%%%%%%%%%%%%%%%%%%%%%%%%

CREOLE resources exhibit a variety of forms depending on the perspective they are
viewed from. Their implementation is as a Java class plus an XML metadata file
living at the same URL. When using GATE Developer, resources can be loaded and
viewed via the resources tree (left pane) and the `create resource' mechanism.
When programming with GATE Embedded, they are Java objects that are obtained by
making calls to GATE's {\tt Factory} class. These various incarnations are the
phases of a CREOLE resource's `lifecycle'. Depending on what sort of task you are
using GATE for, you may use resources in any or all of these phases. For example,
you may only be interested in getting a graphical view of what GATE's ANNIE
Information Extraction system (see \Chapthing\ \ref{chap:annie}) does; in this
case you will use GATE Developer to load the ANNIE resources, and load a
document, and create an ANNIE application and run it on the document. If, on the
other hand, you want to create your own resources, or modify the Java code of an
existing resource (as opposed to just modifying its grammar, for example), you
will need to deal with all the lifecycle phases.

The various phases may be summarised as:
%
\begin{description}
%
\item[Creating a new resource from scratch (bootstrapping).]
To create the binary image of a resource (a Java class in a JAR file),
and the XML file that describes the resource to GATE, you need to
create the appropriate {\tt .java} file(s), compile them and package
them as a {\tt .jar}. GATE provides a bootstrap tool to
start this process -- see
Section \ref{sec:api:bootstrap}. Alternatively you can simply copy
code from an existing resource.
%
\item[Instantiating a resource in GATE Embedded.]
To create a resource in your own Java code, use GATE's {\tt Factory} class
(this takes care of parameterising the resource, restoring it from a database
where appropriate, etc. etc.). Section \ref{sec:api:factory} describes how
to do this.
%
\item[Loading a resource into GATE Developer.]
To load a resource into GATE Developer, use the various `New
... resource' options from the {\tt File} menu and elsewhere. See
Section
\ref{sec:developer:gui}.
%
\item[Resource configuration and implementation.]
GATE's bootstrap tool will create an empty resource that does nothing. In
order to achieve the behaviour you require, you'll need to change the
configuration of the resource (by editing the {\tt creole.xml} file) and/or
change the Java code that implements the resource. See section
\ref{sec:creole-model:config}.
%
\end{description}
%
%More details of the specifics of tasks related to these phases
%are available in \chapthing\ \ref{chap:howto}.




%%%%%%%%%%%%%%%%%%%%%%%%%%%%%%%%%%%%%%%%%%%%%%%%%%%%%%%%%%%%%%%%%%%%%%%%%%%%%
\sect[sec:creole-model:applications]{Processing Resources and Applications}
%%%%%%%%%%%%%%%%%%%%%%%%%%%%%%%%%%%%%%%%%%%%%%%%%%%%%%%%%%%%%%%%%%%%%%%%%%%%%

\mbox{ }

PRs can be combined into {\em applications}. Applications model a control
strategy for the execution of PRs. In GATE, applications are called
`controllers' accordingly.

Currently only sequential, or pipeline,
execution is supported. There are two main types of pipeline:
%
\begin{description}
%
\item[Simple pipelines] simply group a set of PRs together in order and
execute them in turn. The implementing class is called {\tt
SerialController}.
%
\item[Corpus pipelines] are specific for LanguageAnalysers -- PRs that are
applied to documents and corpora. A corpus pipeline opens each document in
the corpus in turn, sets that document as a runtime parameter on each PR,
runs all the PRs on the corpus, then closes the document.
The implementing class is called {\tt SerialAnalyserController}.
\end{description}

Conditional versions of these controllers are also available. These allow
processing resources to be run conditionally on document features. 
See Section \ref{sec:developer:cond} for how to use these.  If more flexibility
is required, the Groovy plugin provides a \emph{scriptable} controller (see
section~\ref{sec:api:groovy:controller}) whose execution strategy is specified
using the Groovy programming language.

Controllers are themselves PRs -- in particular a simple pipeline is a standard
PR and a corpus pipeline is a LanguageAnalyser -- so one pipeline can be nested
in another.  This is particularly useful with conditional controllers to group
together a set of PRs that can all be turned on or off as a group.

There is also a real-time version of the corpus pipeline. When creating such
a controller, a {\tt timeout} parameter needs to be set which determines the
maximum amount of time (in milliseconds) allowed for the processing of a
document. Documents that take longer to process, are simply ignored and the
execution moves to the next document after the timeout interval has lapsed.

All controllers have special handling for processing resources that implement
the interface \texttt{gate.creole.ControllerAwarePR}.  This interface provides
methods that are called by the controller at the start and end of the whole
application's execution -- for a corpus pipeline, this means before any
document has been processed and after all documents in the corpus have been
processed, which is useful for PRs that need to share data structures across
the whole corpus, build aggregate statistics, etc.  For full details, see the
\htlink{http://gate.ac.uk/gate/doc/javadoc/gate/creole/ControllerAwarePR.html}{JavaDoc documentation}
for \texttt{ControllerAwarePR}.
%


%%%%%%%%%%%%%%%%%%%%%%%%%%%%%%%%%%%%%%%%%%%%%%%%%%%%%%%%%%%%%%%%%%%%%%%%%%%%%
\sect[sec:creole-model:datastores]{Language Resources and Datastores}
%%%%%%%%%%%%%%%%%%%%%%%%%%%%%%%%%%%%%%%%%%%%%%%%%%%%%%%%%%%%%%%%%%%%%%%%%%%%%

\mbox{ }

Language Resources can be stored in Datastores. Datastores are an abstract
model of disk-based persistence, which can be implemented by various types of
storage mechanism. Here are the types implemented:
%
\begin{description}
%
\item[Serial Datastores] are based on Java's serialisation system, and store
data directly into files and directories.
%
\item[Lucene Datastores] is a full-featured annotation indexing and
retrieval system. It is provided as part of an extension of the Serial
Datastores. See Section \ref{chap:annic} for more details.
%
%\item[Oracle Datastores] store data into an Oracle RDBMS.
%For details of how to set up an Oracle DB for GATE, see
%\htlinkplain{http://gate.ac.uk/gate/doc/persistence.pdf}.
%
%\item[PostgreSQL Datastores] store data into a PostgreSQL RDBMS.
%For details of how to set up a PostgreSQL DB for GATE, see 
%\htlinkplain{http://gate.ac.uk/gate/doc/persistence.pdf}.
\end{description}

%%%%%%%%%%%%%%%%%%%%%%%%%%%%%%%%%%%%%%%%%%%%%%%%%%%%%%%%%%%%%%%%%%%%%%%%%%%%%
\sect[sec:creole-model:builtins]{Built-in CREOLE Resources}
%%%%%%%%%%%%%%%%%%%%%%%%%%%%%%%%%%%%%%%%%%%%%%%%%%%%%%%%%%%%%%%%%%%%%%%%%%%%%

\mbox{ }

GATE comes with various built-in components:
%
\begin{itemize}
%
\item
Language Resources modelling Documents and Corpora, and various types of
Annotation Schema -- see \Chapthing\ \ref{chap:corpora}.
%
\item
Processing Resources that are part of the ANNIE system -- see \Chapthing\
\ref{chap:annie}.
%
%\item
%Visual Resources for viewing and editing corpora, annotations, etc.
% -- see \chapthing\ \ref{chap:gazetteers}.
%
\item 
Gazetteers -- see \Chapthing\
\ref{chap:gazetteers}.
\item
Ontologies -- see \Chapthing\
\ref{chap:ontologies}.
\item
Machine Learning resources -- see \Chapthing\
\ref{chap:ml}.
\item
Alignment tools -- see \Chapthing\
\ref{chap:alignment}.
\item
Parsers and taggers -- see \Chapthing\
\ref{chap:parsers}.
\item
Other miscellaneous resources -- see \Chapthing\ \ref{chap:misc-creole}.
\end{itemize}

%%%%%%%%%%%%%%%%%%%%%%%%%%%%%%%%%%%%%%%%%%%%%%%%%%%%%%%%%%%%%%%%%%%%%%%%%%%%%
\sect[sec:creole-model:config]{CREOLE Resource Configuration}
%%%%%%%%%%%%%%%%%%%%%%%%%%%%%%%%%%%%%%%%%%%%%%%%%%%%%%%%%%%%%%%%%%%%%%%%%%%%%

This section describes how to supply GATE with the configuration data it needs
about a resource, such as what its parameters are, how to display it if it has
a visualisation, etc.  Several GATE resources can be grouped into a single
\emph{plugin}, which is a directory containing an XML configuration file called
{\tt creole.xml}.  Configuration data for the plugin's resources can be given
in the {\tt creole.xml} file or directly in the Java source file using Java
annotations.

A {\tt creole.xml} file has a root element \verb|<CREOLE-DIRECTORY>|. Traditionally
this element didn't contain any attributes, but with the introduction of installable
plugins (see Sections \ref{sec:developer:installplugins} and
\ref{sec:development:pluginrepository}) the following attributes can now be provided.

\begin{description}
\item[{\tt ID}:] A string that uniquely identifies this plugin. This should
be formatted in a similar way to fully specified Java class names. The class portion
(i.e. everything after the last dot) will be used as the name of the plugin in the
GUI. For example, the obsolete RASP plugin could have the ID gate.obsolete.RASP.
Note that unlike Java class names the plugin name can contain spaces for the purpose
of presentation.
\item[{\tt VERSION}:] The version number of the plugin. For example, 3, 3.1,
3.11, 3.12-SNAPSHOT etc.
\item[{\tt DESCRIPTION}:] A short description of the resources provided by the plugin.
Note that there is really only space for a single sentence in the GUI.
\item[{\tt HELPURL}:] The URL of a web page giving more details about this plugin.
\item[{\tt GATE-MIN}:] The  earliest version of GATE that this plugin is compatible
with. This should be in the same format as the version shown in the GATE titlebar, i.e.
6.1 or 6.2-SNAPSHOT. Do not include the build number information.
\item[{\tt GATE-MAX}:] The last version of GATE which the plugin is compatible with.
This should be in the same format as GATE-MIN.
\end{description}

Currently all these attributes are optional, unless you intend to make the plugin
available through a plugin repository (see Section~\ref{sec:development:pluginrepository}),
in which case the {\tt ID} and {\tt VERSION} attributes must be provided. We would,
however, suggest that developers start to add these attributes to all the plugins
they develop as the information is likely to be used in more places throughout GATE
developer and embeded in the future.

Child elements of the \verb|<CREOLE-DIRECTORY>| depend on the configuration
style.  The following three sections discuss the different styles -- all-XML,
all-annotations and a mixture of the two.

%%%%%%%%%%%%%%%%%%%%%%%%%%%%%%%%%%%%%%%%%%%%%%%%%%%%%%%%%%%%%%%%%%%%%%%%%%%%%
\subsect[sec:creole-model:config:xml]{Configuration with XML}
%%%%%%%%%%%%%%%%%%%%%%%%%%%%%%%%%%%%%%%%%%%%%%%%%%%%%%%%%%%%%%%%%%%%%%%%%%%%%

To configure your resources in the {\tt creole.xml} file, the
\verb|<CREOLE-DIRECTORY>| element should contain one \verb|<RESOURCE>| element
for each resource type in the plugin.  The \verb|<RESOURCE>| elements may
optionally be contained within a \verb|<CREOLE>| element (to allow a single
{\tt creole.xml} file to be built up by concatenating multiple separate files).
For example:

\begin{small}\begin{verbatim}
<CREOLE-DIRECTORY>

<CREOLE>
  <RESOURCE>
    <NAME>Minipar Wrapper</NAME>
    <JAR>MiniparWrapper.jar</JAR>
    <CLASS>minipar.Minipar</CLASS>
    <COMMENT>MiniPar is a shallow parser. It determines the
    dependency relationships between the words of a sentence.</COMMENT>
    <HELPURL>http://gate.ac.uk/cgi-bin/userguide/sec:parsers:minipar</HELPURL>
    <PARAMETER NAME="document"
	  RUNTIME="true"
	  COMMENT="document to process">gate.Document</PARAMETER>
    <PARAMETER NAME="miniparDataDir" 
        RUNTIME="true"
        COMMENT="location of the Minipar data directory">
        java.net.URL
    </PARAMETER>
    <PARAMETER NAME="miniparBinary" 
        RUNTIME="true"
        COMMENT="Name of the Minipar command file">
        java.net.URL
    </PARAMETER>
    <PARAMETER NAME="annotationInputSetName" 
        RUNTIME="true"
        OPTIONAL="true"
        COMMENT="Name of the input Source">
        java.lang.String
    </PARAMETER>
    <PARAMETER NAME="annotationOutputSetName" 
        RUNTIME="true"
        OPTIONAL="true"
        COMMENT="Name of the output AnnotationSetName">
        java.lang.String
    </PARAMETER>
    <PARAMETER NAME="annotationTypeName" 
        RUNTIME="false"
        DEFAULT="DepTreeNode" 
        COMMENT="Annotations to store with this type">
        java.lang.String
    </PARAMETER>
  </RESOURCE>
</CREOLE>
</CREOLE-DIRECTORY>
\end{verbatim}\end{small}

%%%%%%%%%%%%%%%%%%%%%%%%%%%%%%%%%%%%%%%%%%%%%%%%%%%%%%%%%%%%%%%%%%%%%%%%%%%%%
\subsubsect{Basic Resource-Level Data}
%%%%%%%%%%%%%%%%%%%%%%%%%%%%%%%%%%%%%%%%%%%%%%%%%%%%%%%%%%%%%%%%%%%%%%%%%%%%%

Each resource must give a name, a Java class and the JAR file that it can be
loaded from.
The above example is taken from the {\tt Parser\_Minipar} plugin, and defines a
single resource with a number of parameters.

The full list of valid elements under \verb|<RESOURCE>| is as follows:

\begin{description}
\item[NAME] the name of the resource, as it will appear in the `New'
  menu in GATE Developer.  If omitted, defaults to the bare name of
  the resource class (without a package name).
\item[CLASS] the fully qualified name of the Java class that implements this
  resource.
\item[JAR] names JAR files required by this resource (paths are relative to the
  location of {\tt creole.xml}).  Typically this will be the JAR file
  containing the class named by the \verb|<CLASS>| element, but additional
  \verb|<JAR>| elements can be used to name third-party JAR files that the
  resource depends on.
\item[COMMENT] a descriptive comment about the resource, which will
  appear as the tooltip when hovering over an instance of this resource in the
  resources tree in GATE Developer.  If omitted, no comment is used.
\item[HELPURL] a URL to a help document on the web for this resource. It is
  used in the help browser inside GATE Developer.
\item[INTERFACE] the interface type implemented by this resource, for example
  new types of document would specify \verb|<INTERFACE>gate.Document</INTERFACE>|.
\item[ICON] the icon used to represent this resource in GATE Developer.  This is
  a path inside the plugin's JAR file, for
  example \verb|<ICON>/some/package/icon.png</ICON>|.  If the path
  specified does not start with a forward slash, it is assumed to name
  an icon from the GATE default set, which is located in gate.jar at
  gate/resources/img.  If no icon is specified, a generic language
  resource or processing resource icon (as appropriate) is used.
\item[PRIVATE] if present, this resource type is hidden in the GATE Developer GUI, i.e. it is
  not shown in the `New' menus.  This is useful for resource types
  that are intended to be created internally by other resources, or
  for resources that have parameters of a type that cannot be set in
  the GUI.  \verb|<PRIVATE/>| resources can still be created in Java
  code using the {\tt Factory}.
\item[AUTOINSTANCE (and HIDDEN-AUTOINSTANCE)] tells GATE to automatically
  create instances of this resource when the plugin is loaded.  Any number of
  auto instances may be defined, GATE will create them all.  Each
  \verb|<AUTOINSTANCE>| element may optionally contain
  \verb|<PARAM NAME="..." VALUE="..." />| elements giving parameter values to
  use when creating the instance.  Any parameters not specified explicitly will
  take their default values.  Use \verb|<HIDDEN-AUTOINSTANCE>| if you want the
  auto instances not to show up in GATE Developer -- this is useful for things
  like document formats where there should only ever be a single instance in
  GATE and that instance should not be deleted.
\item[TOOL] if present, this resource type is considered to be a ``tool''.
  Tools can contribute items to the Tools menu in GATE Developer.
\end{description}

For visual resources, a \verb|<GUI>| element should also be provided.
This takes a \verb|TYPE| attribute, which can have the value {\tt
LARGE} or {\tt SMALL}.  LARGE means that the visual resource is a
large viewer and should appear in the main part of the GATE Developer
window on the right hand side, SMALL means the VR is a small viewer
which appears in the space below the resources tree in the bottom
left.  The \verb|<GUI>| element supports the following sub-elements:

\begin{description}
\item[RESOURCE\_DISPLAYED] the type of GATE resource this VR can display.  Any
  resource whose type is assignable to this type will be displayed with this
  viewer, so for example a VR that can display all types of document would
  specify \verb|gate.Document|, whereas a VR that can only display the default
  GATE document implementation would specify \verb|gate.corpora.DocumentImpl|.
\item[MAIN\_VIEWER] if present, GATE will consider this VR to be the `most
  important' viewer for the given resource type, and will ensure that if
  several different viewers are all applicable to this resource, this viewer
  will be the one that is initially visible.
\end{description}

For annotation viewers, you should specify an
\verb|<ANNOTATION_TYPE_DISPLAYED>| element giving the annotation type that the
viewer can display (e.g. {\tt Sentence}).

%%%%%%%%%%%%%%%%%%%%%%%%%%%%%%%%%%%%%%%%%%%%%%%%%%%%%%%%%%%%%%%%%%%%%%%%%%%%%
\subsubsect{Resource Parameters}
%%%%%%%%%%%%%%%%%%%%%%%%%%%%%%%%%%%%%%%%%%%%%%%%%%%%%%%%%%%%%%%%%%%%%%%%%%%%%

Resources may also have parameters of various types.
These resources, from the GATE distribution, illustrate the various types of
parameters:
\begin{small}\begin{verbatim}
<RESOURCE>
  <NAME>GATE document</NAME>
  <CLASS>gate.corpora.DocumentImpl</CLASS>
  <INTERFACE>gate.Document</INTERFACE>
  <COMMENT>GATE transient document</COMMENT>
  <OR>
    <PARAMETER NAME="sourceUrl"
      SUFFIXES="txt;text;xml;xhtm;xhtml;html;htm;sgml;sgm;mail;email;eml;rtf"
      COMMENT="Source URL">java.net.URL</PARAMETER>
    <PARAMETER NAME="stringContent"
      COMMENT="The content of the document">java.lang.String</PARAMETER>
  </OR>
  <PARAMETER
    COMMENT="Should the document read the original markup"
    NAME="markupAware" DEFAULT="true">java.lang.Boolean</PARAMETER>
  <PARAMETER NAME="encoding" OPTIONAL="true"
    COMMENT="Encoding" DEFAULT="">java.lang.String</PARAMETER>
  <PARAMETER NAME="sourceUrlStartOffset"
    COMMENT="Start offset for documents based on ranges"
    OPTIONAL="true">java.lang.Long</PARAMETER>
  <PARAMETER NAME="sourceUrlEndOffset"
    COMMENT="End offset for documents based on ranges"
    OPTIONAL="true">java.lang.Long</PARAMETER>
  <PARAMETER NAME="preserveOriginalContent"
    COMMENT="Should the document preserve the original content"
    DEFAULT="false">java.lang.Boolean</PARAMETER>
  <PARAMETER NAME="collectRepositioningInfo"
    COMMENT="Should the document collect repositioning information"
    DEFAULT="false">java.lang.Boolean</PARAMETER>
  <ICON>lr.gif</ICON>
</RESOURCE>
\end{verbatim}\end{small}

\begin{small}\begin{verbatim}
<RESOURCE>
  <NAME>Document Reset PR</NAME>
  <CLASS>gate.creole.annotdelete.AnnotationDeletePR</CLASS>
  <COMMENT>Document cleaner</COMMENT>
  <PARAMETER NAME="document" RUNTIME="true">gate.Document</PARAMETER>
  <PARAMETER NAME="annotationTypes" RUNTIME="true"
    OPTIONAL="true">java.util.ArrayList</PARAMETER>
</RESOURCE>
\end{verbatim}\end{small}

Parameters may be optional, and may have default values (and may have
comments to describe their purpose, which is displayed by GATE
Developer during interactive parameter setting).

Some PR parameters are execution time ({\tt RUNTIME}),
some are initialisation time.
E.g. at execution time a doc is supplied to a language analyser;
at initialisation time a grammar may be supplied to a language analyser.

The \verb|<PARAMETER>| tag takes the following attributes:
\begin{description}
\item[NAME:]
name of the JavaBean property that the parameter refers to, i.e. for a
parameter named `someParam' the class must have {\tt setSomeParam} and
{\tt getSomeParam} methods.\footnote{The JavaBeans spec allows {\tt is} instead
of {\tt get} for properties of the primitive type {\tt boolean}, but GATE does
not support parameters with primitive types.  Parameters of type
{\tt java.lang.Boolean} (the wrapper class) are permitted, but these have
{\tt get} accessors anyway.}
\item[DEFAULT:]
default value (see below).
\item[RUNTIME:]
doesn't need setting at initialisation time, but must be set before calling
{\tt execute()}. Only meaningful for PRs
\item[OPTIONAL:]
not required
\item[COMMENT:]
for display purposes
\item[ITEM\_CLASS\_NAME:]
(only applies to parameters whose type is {\tt java.util.Collection}
or a type that implements or extends this) this specifies the type of
elements the collection contains, so GATE can use the right
type when parameters are set.  If omitted, GATE will pass in
the elements as Strings.
\item[SUFFIXES:]
(only applies to parameters of type {\tt java.net.URL}) a
semicolon-separated list of file suffixes that this parameter
typically accepts, used as a filter in the file chooser provided by
GATE Developer to select a local file as the parameter value.
\end{description}

It is possible for two or more parameters to be mutually exclusive (i.e. a user
must specify one or the other but not both).  In this case the
\verb|<PARAMETER>| elements should be grouped together under an \verb|<OR>|
element.

The type of the parameter is specified as the text of
the \verb|<PARAMETER>| element, and the type supplied must match the
return type of the parameter's {\tt get} method.  Any reference type
(class, interface or enum) may be used as the parameter type,
including other resource types -- in this case GATE Developer will
offer a list of the loaded instances of that resource as options for
the parameter value.  Primitive types (char, boolean, \ldots) are not
supported, instead you should use the corresponding wrapper type ({\tt
java.lang.Character}, {\tt java.lang.Boolean}, \ldots).  If the getter
returns a parameterized type (e.g. \verb|List<Integer>|) you should
just specify the raw type ({\tt java.util.List}) here\footnote{In this
particular case, as the type is a collection, you would specify {\tt
java.lang.Integer} as the {\tt ITEM\_CLASS\_NAME}.}.

The DEFAULT string is converted to the appropriate type for the parameter -
\texttt{java.lang.String} parameters use the value directly, primitive wrapper
types e.g. \texttt{java.lang.Integer} use their respective \texttt{valueOf}
methods, and other built-in Java types can have defaults specified provided
they have a constructor taking a \texttt{String}.

The type \texttt{java.net.URL} is treated specially: if the default
string is not an absolute URL (e.g. http://gate.ac.uk/) then it is treated as a
path relative to the location of the \texttt{creole.xml} file.  Thus a DEFAULT
of \texttt{`resources/main.jape'} in the file
\texttt{file:/opt/MyPlugin/creole.xml} is treated as the absolute URL
\texttt{file:/opt/MyPlugin/resources/main.jape}.

For \texttt{Collection}-valued parameters multiple values may be specified,
separated by semicolons, e.g. \texttt{`foo;bar;baz'}; if the parameter's type
is an interface -- \texttt{Collection} or one of its sub-interfaces (e.g.
\texttt{List}) -- a suitable concrete class (e.g. \texttt{ArrayList},
\texttt{HashSet}) will be chosen automatically for the default value.

For parameters of type \texttt{gate.FeatureMap} multiple \texttt{name=value}
pairs can be specified, e.g.  \texttt{`kind=word;orth=upperInitial'}.  For
\texttt{enum}-valued parameters the default string is taken as the name of the
enum constant to use.  Finally, if no DEFAULT attribute is specified, the
default value is \texttt{null}.

%%%%%%%%%%%%%%%%%%%%%%%%%%%%%%%%%%%%%%%%%%%%%%%%%%%%%%%%%%%%%%%%%%%%%%%%%%%%%
\subsect[sec:creole-model:config:annotations]{Configuring Resources using Annotations}
%%%%%%%%%%%%%%%%%%%%%%%%%%%%%%%%%%%%%%%%%%%%%%%%%%%%%%%%%%%%%%%%%%%%%%%%%%%%%

As an alternative to the XML configuration style, GATE provides Java
annotation types to embed the configuration data directly in the Java source
code.  \verb|@CreoleResource| is used to mark a class as a GATE resource, and
parameter information is provided through annotations on the JavaBean {\tt set}
methods.  At runtime these annotations are read and mapped into the equivalent
entries in {\tt creole.xml} before parsing.  The metadata annotation types are
all marked \verb|@Documented| so the CREOLE configuration data will be visible
in the generated JavaDoc documentation.

For more detailed information, see the
\htlink{http://gate.ac.uk/gate/doc/javadoc/gate/creole/metadata/package-summary.html}{JavaDoc documentation for {\tt gate.creole.metadata}}.

To use annotation-driven configuration for a plugin a {\tt creole.xml} file is
still required but it need only contain the following:
\begin{small}\begin{verbatim}
<CREOLE-DIRECTORY>
  <JAR SCAN="true">myPlugin.jar</JAR>
  <JAR>lib/thirdPartyLib.jar</JAR>
</CREOLE-DIRECTORY>
\end{verbatim}\end{small}

This tells GATE to load {\tt myPlugin.jar} and scan its contents looking for
resource classes annotated with \verb|@CreoleResource|.  Other JAR files
required by the plugin can be specified using other \verb|<JAR>| elements
without \verb|SCAN="true"|.

In a GATE Embedded application it is possible to register a single
\verb|@CreoleResource| annotated class without using a {\tt creole.xml} file
by calling
\begin{small}\begin{verbatim}
Gate.getCreoleRegister().registerComponent(MyResource.class);
\end{verbatim}\end{small}
%
GATE will extract the configuration from the annotations on the class and make
it available for use as if it had been defined in a plugin.

%%%%%%%%%%%%%%%%%%%%%%%%%%%%%%%%%%%%%%%%%%%%%%%%%%%%%%%%%%%%%%%%%%%%%%%%%%%%%
\subsubsect{Basic Resource-Level Data}
%%%%%%%%%%%%%%%%%%%%%%%%%%%%%%%%%%%%%%%%%%%%%%%%%%%%%%%%%%%%%%%%%%%%%%%%%%%%%

To mark a class as a CREOLE resource, simply use the \verb|@CreoleResource|
annotation (in the {\tt gate.creole.metadata} package), for example:

\begin{lstlisting}
import gate.creole.AbstractLanguageAnalyser;
import gate.creole.metadata.*;

@CreoleResource(name = "GATE Tokeniser",
                comment = "Splits text into tokens and spaces")
public class Tokeniser extends AbstractLanguageAnalyser {
  ...
\end{lstlisting}

The \verb|@CreoleResource| annotation provides slots for all the values that
can be specified under \verb|<RESOURCE>| in {\tt creole.xml}, except
\verb|<CLASS>| (inferred from the name of the annotated class) and \verb|<JAR>|
(taken to be the JAR containing the class):
\begin{description}
\item[name] (String) the name of the resource, as it will appear in the `New'
  menu in GATE Developer.  If omitted, defaults to the bare name of the resource
  class (without a package name). (XML equivalent \verb|<NAME>|)
\item[comment] (String) a descriptive comment about the resource, which will
  appear as the tooltip when hovering over an instance of this
  resource in the resources tree in GATE Developer.  If omitted, no
  comment is used. (XML equivalent \verb|<COMMENT>|)
\item[helpURL] (String) a URL to a help document on the web for this
  resource. It is used in the help browser inside GATE Developer. (XML
  equivalent \verb|<HELPURL>|)
\item[isPrivate] (boolean) should this resource type be hidden from the GATE Developer GUI, so
  it does not appear in the `New' menus?  If omitted, defaults to
  false (i.e.  not hidden). (XML equivalent \verb|<PRIVATE/>|)
\item[icon] (String) the icon to use to represent the resource in GATE Developer.
  If omitted, a generic language resource or processing resource icon
  is used.  (XML equivalent \verb|<ICON>|, see the description above
  for details)
\item[interfaceName] (String) the interface type implemented by this resource,
  for example a new type of document would specify \verb|"gate.Document"| here.
  (XML equivalent \verb|<INTERFACE>|)
\item[autoInstances] (array of {\tt @AutoInstance} annotations) definitions for
  any instances of this resource that should be created automatically when the
  plugin is loaded.  If omitted, no auto-instances are created by default. (XML
  equivalent, one or more \verb|<AUTOINSTANCE>| and/or
  \verb|<HIDDEN-AUTOINSTANCE>| elements, see the description above for details)
\item[tool] (boolean) is this resource type a tool?
\end{description}

For visual resources only, the following elements are also available:
\begin{description}
\item[guiType] (GuiType enum) the type of GUI this resource defines.
  (XML equivalent \verb~<GUI TYPE="LARGE|SMALL">~)
\item[resourceDisplayed] (String) the class name of the resource type that this
  VR displays, e.g. \verb|"gate.Corpus"|. (XML equivalent
  \verb|<RESOURCE_DISPLAYED>|)
\item[mainViewer] (boolean) is this VR the `most important' viewer for its
  displayed resource type? (XML equivalent \verb|<MAIN_VIEWER/>|, see above for
  details)
\end{description}

For annotation viewers, you should specify an
\verb|annotationTypeDisplayed| element giving the annotation type that the
viewer can display (e.g. {\tt Sentence}).

%%%%%%%%%%%%%%%%%%%%%%%%%%%%%%%%%%%%%%%%%%%%%%%%%%%%%%%%%%%%%%%%%%%%%%%%%%%%%
\subsubsect{Resource Parameters}
%%%%%%%%%%%%%%%%%%%%%%%%%%%%%%%%%%%%%%%%%%%%%%%%%%%%%%%%%%%%%%%%%%%%%%%%%%%%%

Parameters are declared by placing annotations on their JavaBean {\tt set}
methods.  To mark a setter method as a parameter, use the
\verb|@CreoleParameter| annotation, for example:

\begin{small}\begin{verbatim}
  @CreoleParameter(comment = "The location of the list of abbreviations")
  public void setAbbrListUrl(URL listUrl) {
    ...
\end{verbatim}\end{small}

GATE will infer the parameter's name from the name of the JavaBean property in
the usual way (i.e. strip off the leading {\tt set} and convert the following
character to lower case, so in this example the name is {\tt abbrListUrl}).
The parameter name is \emph{not} taken from the name of the method parameter.
The parameter's type is inferred from the type of the method parameter
({\tt java.net.URL} in this case).

The annotation elements of \verb|@CreoleParameter| correspond to the attributes
of the \verb|<PARAMETER>| tag in the XML configuration style:
\begin{description}
\item[comment] (String) an optional descriptive comment about the parameter.
  (XML equivalent \verb|COMMENT|)
\item[defaultValue] (String) the optional default value for this parameter.
  The value is specified as a string but is converted to the relevant type by
  GATE according to the conversions described in the previous section.  Note
  that relative path default values for URL-valued parameters are still
  relative to the location of the {\tt creole.xml} file, not the annotated
  class\footnote{When registering a class using {\tt
  CreoleRegister.registerComponent} the base URL against which defaults for URL
  parameters are resolved is not specified.  In such a resource it may be
  better to use {\tt Class.getResource} to construct the default URLs if no
  value is supplied for the parameter by the user.}.  (XML equivalent
  \verb|DEFAULT|)
\item[suffixes] (String) for URL-valued parameters, a semicolon-separated list
  of default file suffixes that this parameter accepts. (XML equivalent
  \verb|SUFFIXES|)
\item[collectionElementType] (Class) for {\tt Collection}-valued parameters,
  the type of the elements in the collection.  This can usually be inferred
  from the generic type information, for example
  \verb|public void setIndices(List<Integer> indices)|, but must be specified
  if the {\tt set} method's parameter has a raw (non-parameterized) type.
  (XML equivalent \verb|ITEM_CLASS_NAME|)
\end{description}

Mutually-exclusive parameters (such as would be grouped in an \verb|<OR>| in
{\tt creole.xml}) are handled by adding a {\tt disjunction="{\it label}"} and
{\tt priority={\it n}} to the \verb|@CreoleParameter| annotation -- all
parameters that share the same label are grouped in the same disjunction, and
will be offered in order of priority.  The parameter with the smallest priority
value will be the one listed first, and thus the one that is offered initially
when creating a resource of this type in GATE Developer.  For example, the
following is a simplified extract from {\tt gate.corpora.DocumentImpl}:

\begin{lstlisting}
@CreoleParameter(disjunction="src", priority=1)
public void setSourceUrl(URL src) { /* */ }

@CreoleParameter(disjunction="src", priority=2)
public void setStringContent(String content) { /* */ }
\end{lstlisting}

This declares the parameters ``stringContent'' and ``sourceUrl'' as
mutually-exclusive, and when creating an instance of this resource in GATE
Developer the parameter that will be shown initially is sourceUrl.  To set
stringContent instead the user must select it from the drop-down list.
Parameters with the same declared priority value will appear next to each other
in the list, but their relative ordering is not specified.  Parameters with no
explicit priority are always listed {\it after} those that do specify a
priority.

Optional and runtime parameters are marked using extra annotations, for example:
\begin{lstlisting}
  @Optional
  @RunTime
  @CreoleParameter
  public void setAnnotationSetName(String asName) {
    ...
\end{lstlisting}

%%%%%%%%%%%%%%%%%%%%%%%%%%%%%%%%%%%%%%%%%%%%%%%%%%%%%%%%%%%%%%%%%%%%%%%%%%%%%
\subsubsect{Inheritance}
%%%%%%%%%%%%%%%%%%%%%%%%%%%%%%%%%%%%%%%%%%%%%%%%%%%%%%%%%%%%%%%%%%%%%%%%%%%%%

Unlike with pure XML configuration, when using annotations a resource
will inherit any configuration data that was not explicitly specified
from annotations on its parent class and on any interfaces it
implements.  Specifically, if you do not specify a comment,
interfaceName, icon, annotationTypeDisplayed or the GUI-related
elements (guiType and resourceDisplayed) on
your \verb|@CreoleResource| annotation then GATE will look up the
class tree for other \verb|@CreoleResource| annotations, first on the
superclass, its superclass, etc., then at any implemented interfaces,
and use the first value it finds.  This is useful if you are defining
a family of related resources that inherit from a common base class.

The resource name and the {\tt isPrivate} and {\tt mainViewer} flags are
\emph{not} inherited.

Parameter definitions are inherited in a similar way.  This is one of the big
advantages of annotation configuration over pure XML -- if one resource class
extends another then with pure XML configuration all the parent class's
parameter definitions must be duplicated in the subclass's {\tt creole.xml}
definition.  With annotations, parameters are inherited from the parent class
(and its parent, etc.) as well as from any interfaces implemented.  For
example, the {\tt gate.LanguageAnalyser} interface provides two parameter
definitions via annotated {\tt set} methods, for the {\tt corpus} and
{\tt document} parameters.  Any \verb|@CreoleResource| annotated class that
implements {\tt LanguageAnalyser}, directly or indirectly, will get these
parameters automatically.

Of course, there are some cases where this behaviour is not desirable, for
example if a subclass calculates a value for a superclass parameter rather than
having the user set it directly.  In this case you can hide the parameter by
overriding the {\tt set} method in the subclass and using a marker annotation:
\begin{lstlisting}
  @HiddenCreoleParameter
  public void setSomeParam(String someParam) {
    super.setSomeParam(someParam);
  }
\end{lstlisting}

The overriding method will typically just call the superclass one, as its only
purpose is to provide a place to put the \verb|@HiddenCreoleParameter|
annotation.

Alternatively, you may want to override some of the configuration for a
parameter but inherit the rest from the superclass.  Again, this is handled by
trivially overriding the {\tt set} method and re-annotating it:
\begin{lstlisting}
  // superclass
  @CreoleParameter(comment = "Location of the grammar file",
                   suffixes = "jape")
  public void setGrammarUrl(URL grammarLocation) {
    ...
  }

  @Optional
  @RunTime
  @CreoleParameter(comment = "Feature to set on success")
  public void setSuccessFeature(String name) {
    ...
  }
\end{lstlisting}
\begin{lstlisting}
  //-----------------------------------
  // subclass
  
  // override the default value, inherit everything else
  @CreoleParameter(defaultValue = "resources/defaultGrammar.jape")
  public void setGrammarUrl(URL url) {
    super.setGrammarUrl(url);
  }

  // we want the parameter to be required in the subclass
  @Optional(false)
  @CreoleParameter
  public void setSuccessFeature(String name) {
    super.setSuccessFeature(name);
  }
\end{lstlisting}

Note that for backwards compatibility, data is only inherited from superclass
annotations if the subclass is itself annotated with \verb|@CreoleResource|.
If the subclass is not annotated then GATE assumes that \emph{all} its
configuration is contained in {\tt creole.xml} in the usual way.

%%%%%%%%%%%%%%%%%%%%%%%%%%%%%%%%%%%%%%%%%%%%%%%%%%%%%%%%%%%%%%%%%%%%%%%%%%%%%
\subsect[sec:creole-model:config:mixed]{Mixing the Configuration Styles}
%%%%%%%%%%%%%%%%%%%%%%%%%%%%%%%%%%%%%%%%%%%%%%%%%%%%%%%%%%%%%%%%%%%%%%%%%%%%%

It is possible and often useful to mix and match the XML and annotation-driven
configuration styles.  The rule is always that \emph{anything specified in the
XML takes priority over the annotations}.  The following examples show what
this allows.

%%%%%%%%%%%%%%%%%%%%%%%%%%%%%%%%%%%%%%%%%%%%%%%%%%%%%%%%%%%%%%%%%%%%%%%%%%%%%
\subsubsect{Overriding Configuration for a Third-Party Resource}
%%%%%%%%%%%%%%%%%%%%%%%%%%%%%%%%%%%%%%%%%%%%%%%%%%%%%%%%%%%%%%%%%%%%%%%%%%%%%

Suppose you have a plugin from some third party that uses annotation-driven
configuration.  You don't have the source code but you would like to override
the default value for one of the parameters of one of the plugin's resources.
You can do this in the {\tt creole.xml}:
\begin{small}\begin{verbatim}
<CREOLE-DIRECTORY>
  <JAR SCAN="true">acmePlugin-1.0.jar</JAR>

  <!-- Add the following to override the annotations -->
  <RESOURCE>
    <CLASS>com.acme.plugin.UsefulPR</CLASS>
    <PARAMETER NAME="listUrl"
      DEFAULT="resources/myList.txt">java.net.URL</PARAMETER>
  </RESOURCE>
</CREOLE-DIRECTORY>
\end{verbatim}\end{small}

The default value for the {\tt listUrl} parameter in the annotated class will
be replaced by your value.

%%%%%%%%%%%%%%%%%%%%%%%%%%%%%%%%%%%%%%%%%%%%%%%%%%%%%%%%%%%%%%%%%%%%%%%%%%%%%
\subsubsect{External AUTOINSTANCEs}
%%%%%%%%%%%%%%%%%%%%%%%%%%%%%%%%%%%%%%%%%%%%%%%%%%%%%%%%%%%%%%%%%%%%%%%%%%%%%

For resources like document formats, where there should always and only be one
instance in GATE at any time, it makes sense to put the auto-instance
definitions in the \verb|@CreoleResource| annotation.  But if the automatically
created instances are a convenience rather than a necessity it may be better
to define them in XML so other users can disable them without re-compiling the
class:
\begin{small}\begin{verbatim}
<CREOLE-DIRECTORY>
  <JAR SCAN="true">myPlugin.jar</JAR>

  <RESOURCE>
    <CLASS>com.acme.AutoPR</CLASS>
    <AUTOINSTANCE>
      <PARAM NAME="type" VALUE="Sentence" />
    </AUTOINSTANCE>
    <AUTOINSTANCE>
      <PARAM NAME="type" VALUE="Paragraph" />
    </AUTOINSTANCE>
  </RESOURCE>
</CREOLE-DIRECTORY>
\end{verbatim}\end{small}

%%%%%%%%%%%%%%%%%%%%%%%%%%%%%%%%%%%%%%%%%%%%%%%%%%%%%%%%%%%%%%%%%%%%%%%%%%%%%
\subsubsect{Inheriting Parameters}
%%%%%%%%%%%%%%%%%%%%%%%%%%%%%%%%%%%%%%%%%%%%%%%%%%%%%%%%%%%%%%%%%%%%%%%%%%%%%

If you would prefer to use XML configuration for your own resources, but would
like to benefit from the parameter inheritance features of the
annotation-driven approach, you can write a normal {\tt creole.xml} file with
all your configuration and just add a blank \verb|@CreoleResource| annotation
to your class.  For example:
\begin{lstlisting}
package com.acme;
import gate.*;
import gate.creole.metadata.CreoleResource;

@CreoleResource
public class MyPR implements LanguageAnalyser {
  ...
}
\end{lstlisting}

\begin{small}\begin{verbatim}
<!-- creole.xml -->
<CREOLE-DIRECTORY>
  <CREOLE>
    <RESOURCE>
      <NAME>My Processing Resource</NAME>
      <CLASS>com.acme.MyPR</CLASS>
      <COMMENT>...</COMMENT>
      <PARAMETER NAME="annotationSetName"
        RUNTIME="true" OPTIONAL="true">java.lang.String</PARAMETER>
      <!--
      don't need to declare document and corpus parameters, they
      are inherited from LanguageAnalyser
      -->
    </RESOURCE>
  </CREOLE>
</CREOLE-DIRECTORY>
\end{verbatim}\end{small}

N.B. Without the \verb|@CreoleResource| the parameters would not be inherited.

%%%%%%%%%%%%%%%%%%%%%%%%%%%%%%%%%%%%%%%%%%%%%%%%%%%%%%%%%%%%%%%%%%%%%%%%%%%%%
\subsect[sec:creole-model:config:ivy]{Loading Third-Party Libraries using Apache Ivy}
%%%%%%%%%%%%%%%%%%%%%%%%%%%%%%%%%%%%%%%%%%%%%%%%%%%%%%%%%%%%%%%%%%%%%%%%%%%%%

With ``simple'' plugins most of the code is contained in a single jar or relies
on just one or two thrid-party libraries which are easy to enumerate within
{\tt creole.xml} in order for them to be loaded into GATE when the plugin is
loaded. More complex plugins can, however, rely on a large number of third-party
libraries, each of which may have it's own dependencies. In an attempt to
simplify the management of third-party libraries, within CREOLE plugins,
\htlink{http://ant.apache.org/ivy/}{Apache Ivy} can be used to specify the
dependencies.

No attempt is made here to explain the workings of Ivy or the format of
the {\tt ivy.xml} file. For full details you should refer to the
\htlink{http://ant.apache.org/ivy/history/latest-milestone/ivyfile.html}{approprioate 
section of the Ivy manual}.

Incorporating an Ivy file within a CREOLE plugin is as simple as referencing
it from within {\tt creole.xml}. Assumuing you have used the default filename
of {\tt ivy.xml} then you can reference it via a simple \verb|<IVY>| element.

\begin{small}\begin{verbatim}
<CREOLE-DIRECTORY>
  <JAR SCAN="true">myPlugin.jar</JAR>
  <IVY/>
</CREOLE-DIRECTORY>
\end{verbatim}\end{small}

If you have used an alternative filename then you can specify it as the text
content of the \verb|<IVY>| element. For example, if the filename is {\tt plugin-ivy.xml}
you would reference it as follows:

\begin{small}\begin{verbatim}
<CREOLE-DIRECTORY>
  <JAR SCAN="true">myPlugin.jar</JAR>
  <IVY>plugin-ivy.xml</IVY>
</CREOLE-DIRECTORY>
\end{verbatim}\end{small}

When the plugin is loaded into GATE Ivy resolves the dependencies, downloads the
appropriate libraries (if necessary) and then makes them available to the plugin.
Once the plugin is loaded it behaves exactly the same as any other plugin.

Note that if you export an application (see Section \ref{sec:developer:export})
then to ensure that it is self-contained and useable within any processing
environment the Ivy based dependencies are expanded; the libraries are
downloaded into the plugin's lib folder, appropriate entires are added to
{\tt creole.xml} and the \verb|<IVY>| element is removed.

%%%%%%%%%%%%%%%%%%%%%%%%%%%%%%%%%%%%%%%%%%%%%%%%%%%%%%%%%%%%%%%%%%%%%%%%%%%%%
\sect[sec:creole-model:tools]{Tools: How to Add Utilities to GATE Developer}
%%%%%%%%%%%%%%%%%%%%%%%%%%%%%%%%%%%%%%%%%%%%%%%%%%%%%%%%%%%%%%%%%%%%%%%%%%%%%

Visual Resources allow a developer to provide a GUI to interact with a
particular resource type (PR or LR), but sometimes it is useful to provide
general utilities for use in the GATE Developer GUI that are not tied to any
specific resource type.  Examples include the annotation diff tool and the
Groovy console (provided by the \verb|Groovy| plugin), both of which are
self-contained tools that display in their own top-level window.  To support
this, the CREOLE model has the concept of a {\em tool}.

A resource type is marked as a tool by using the \verb|<TOOL/>| element in its
\verb|creole.xml| definition, or by setting \verb|tool = true| if using the
@CreoleResource annotation configuration style.  If a resource is declared to
be a tool, and written to implement the \verb|gate.gui.ActionsPublisher|
interface, then whenever an instance of the resource is created its published
actions will be added to the ``Tools'' menu in GATE Developer.

Since the published actions of {\em every} instance of the resource will be
added to the tools menu, it is best not to use this mechanism on resource types
that can be instantiated by the user.  The ``tool'' marker is best used in
combination with the ``private'' flag (to hide the resource from the list of
available types in the GUI) and one or more hidden autoinstance definitions
to create a limited number of instances of the resource when its defining
plugin is loaded.  See the \verb|GroovySupport| resource in the \verb|Groovy|
plugin for an example of this.

\subsect[sec:creole-model:tools:menu-path]{Putting Your Tools in a Sub-Menu}

If your plugin provides a number of tools (or a number of actions from the same
tool) you may wish to organise your actions into one or more sub-menus, rather
than placing them all on the single top-level tools menu.  To do this, you need
to put a special value into the actions returned by the tool's
\lstinline!getActions()! method:
\begin{lstlisting}
action.putValue(GateConstants.MENU_PATH_KEY,
    new String[] {"Acme toolkit", "Statistics"});
\end{lstlisting}
The key must be \lstinline!GateConstants.MENU_PATH_KEY! and the value must be
an array of strings.  Each string in the array represents the name of one level
of sub-menus.  Thus in the example above the action would be placed under
``Tools $\rightarrow$ Acme toolkit $\rightarrow$ Statistics''.  If no
\lstinline!MENU_PATH_KEY! value is provided the action will be placed directly
on the Tools menu.

\subsect[sec:creole-model:tools:resourcehelpers]{Adding Tools To Existing
Resource Types}

While Visual Resources (VR) allow you to add new features to a particular
resource they have a number of shortcomings. Firstly not every new feature
will require a full VR; often a new entry on the resources right-click
menu will suffice. More importantly new feautres added via a VR are only
available while the VR is open. A Resource Helper is a form of Tool, as
above, which can add new menu options to any existing resource type without
requiring a VR.

A Resource Helper is defined in the same way as a Tool (by setting the
\verb|tool = true| feature of the @CreoleResource annotation and loaded via an
autoinstance definition) but must also extend the \verb|gate.gui.ResourceHelper|
class. A Resource Helper can then return a set of actions for a given resource
which will be added to its right-click menu. See the \verb|FastInfosetExporter|
resource in the \verb|Format_FastInfoset| plugin for an example of how this works.

A Resource Helper may also make new API calls accessable to allow similar
functionality to be made available to GATE Embedded, see Section
\ref{sec:api:resourcehelpers} for more details on how this works.
