%%%%%%%%%%%%%%%%%%%%%%%%%%%%%%%%%%%%%%%%%%%%%%%%%%%%%%%%%%%%%%%%%%%%%%%%%%%%%
%
% creole-model.tex
%
% hamish, 25/8/1
%
% $Id: creole-model.tex,v 1.16 2005/02/10 17:08:59 ian Exp $
%
%%%%%%%%%%%%%%%%%%%%%%%%%%%%%%%%%%%%%%%%%%%%%%%%%%%%%%%%%%%%%%%%%%%%%%%%%%%%%


%%%%%%%%%%%%%%%%%%%%%%%%%%%%%%%%%%%%%%%%%%%%%%%%%%%%%%%%%%%%%%%%%%%%%%%%%%%%%
\chapt[chap:creole-model]{CREOLE: the GATE Component Model}
\markboth{CREOLE: the GATE Component Model}{CREOLE: the GATE Component Model}
%%%%%%%%%%%%%%%%%%%%%%%%%%%%%%%%%%%%%%%%%%%%%%%%%%%%%%%%%%%%%%%%%%%%%%%%%%%%%

%%%% qqqqqqqqqqqqqqqqqqqqqqqqq %%%%
\ifprintedbook
\else
\begin{quote}
\ldots Noam Chomsky's answer in {\em Secrets, Lies and Democracy}
(David Barsamian 1994; Odonian) to `What do you think about the Internet?'

`I think that there are good things about it, but there are also
aspects of it that concern and worry me. This is an intuitive
response -- I can't prove it -- but my feeling is that, since people
aren't Martians or robots, direct face-to-face contact is an
extremely important part of human life. It helps develop self-understanding
and the growth of a healthy personality.

`You just have a different relationship to somebody when
you're looking at them than you do when you're punching away at a
keyboard and some symbols come back. I suspect that
extending that form of abstract and remote relationship, instead of
direct, personal contact, is going to have unpleasant
effects on what people are like. It will diminish their humanity, I think.'

Chomsky, quoted at \htlinkplain{http://philip.greenspun.com/wtr/dead-trees/53015}.
\end{quote}
\fi
%%%% qqqqqqqqqqqqqqqqqqqqqqqqq %%%%

The GATE architecture is based on components: reusable chunks of software
with well-defined interfaces that may be deployed in a variety of contexts.
The design of GATE is based on an analysis of previous work on
infrastructure for LE, and of the typical types of software entities
found in the fields of NLP and CL (see in particular chapters 4--6
of \cite{Cun00a}). Our research suggested that a profitable way
to support LE software development was an architecture that
breaks down such programs into components of various types.
Because LE practice varies very widely (it is, after all, predominantly
a research field), the architecture must avoid restricting the sorts
of components that developers can plug into the infrastructure.
The GATE framework accomplishes this via an adapted
version of the {\em Java Beans} component framework from Sun, as described in
section~\ref{sec:creole-model:framework}.

GATE components may be implemented by a variety of programming languages and
databases, but in each case they are represented to the system as a Java
class. This class may do nothing other than call the underlying program, or
provide an access layer to a database; on the other hand
it may implement the whole component.

GATE components are one of three types:
%
\begin{itemize}
%
\item
LanguageResources (LRs) represent entities such as lexicons, corpora or
ontologies;
%
\item
ProcessingResources (PRs) represent entities that are primarily algorithmic,
such as parsers, generators or ngram modellers;
%
\item
VisualResources (VRs) represent visualisation and editing components that
participate in GUIs.
%
\end{itemize}
%
\ifprintedbook
\else
The distinction between language resources and processing resources is explored more fully in section \ref{sec:design:components}.
\fi
Collectively, the set of resources integrated with GATE is known as
{\bf CREOLE}: a
Collection of REusable Objects for Language Engineering.

In the rest of this \chapthing:
\begin{itemize}
\item
Section \ref{sec:creole-model:lifecycle} describes the lifecycle of GATE
components;
\item
Section \ref{sec:creole-model:applications} describes how Processing Resources
can be grouped into applications;
\item
Section \ref{sec:creole-model:datastores} describes the relationship between
Language Resources and their datastores;
\item
Section \ref{sec:creole-model:builtins} summarises GATE's set of built-in
components;
\item
Section \ref{sec:creole-model:config} describes how configuration data for
Resource types is supplied to GATE.
\end{itemize}


%%%%%%%%%%%%%%%%%%%%%%%%%%%%%%%%%%%%%%%%%%%%%%%%%%%%%%%%%%%%%%%%%%%%%%%%%%%%%
\sect[sec:creole-model:creoleweb]{The Web and CREOLE}
%%%%%%%%%%%%%%%%%%%%%%%%%%%%%%%%%%%%%%%%%%%%%%%%%%%%%%%%%%%%%%%%%%%%%%%%%%%%%

GATE allows resource implementations and Language Resource persistent data to
be distributed over the Web, and uses Java annotations for
configuration of resources (and GATE itself).

Resource implementations are grouped together as `plugins', stored
either in a single JAR file published via the standard Maven repository
mechanism, or at a URL (when the resources are in the local file system this
would be a {\tt file:/} URL). When a plugin is loaded into GATE it looks
for a configuration file called {\tt creole.xml} relative to the plugin URL or
inside the plugin JAR file and uses the contents of this file in combination
with Java annotations on the source code to determine what resources this
plugin declares and, in the case of directory-style plugins, where to find the
classes that implement the resource types (typically a JAR file in the plugin
directory).  GATE retrieves the configuration information from the plugin's
resource classes and adds the resource definitions to the CREOLE register. When
a user requests an instantiation of a resource, GATE creates an instance of the
resource class in the virtual machine.

Language resource data can be stored in binary serialised form in the local
file system.
%, or in an RDBMS like Oracle. In the latter case, communication
%with the database is over JDBC\footnote{The Java DataBase Connectivity
%layer.}, allowing the data to be located anywhere on the network (or anywhere
%you can get Oracle running, that is!).


%%%%%%%%%%%%%%%%%%%%%%%%%%%%%%%%%%%%%%%%%%%%%%%%%%%%%%%%%%%%%%%%%%%%%%%%%%%%%
\sect[sec:creole-model:framework]{The GATE Framework}
\label{sec:creole-model:beans}
%%%%%%%%%%%%%%%%%%%%%%%%%%%%%%%%%%%%%%%%%%%%%%%%%%%%%%%%%%%%%%%%%%%%%%%%%%%%%

We can think of the GATE framework as a backplane into
which users can plug CREOLE components.
The user gives the system a list of plugins to search when it starts up,
and components in those plugins are loaded by the system.

The backplane performs these functions:
\begin{itemize}
\item
component discovery, bootstrapping, loading and reloading;
\item
management and visualisation of
native data structures for common information types;
\item
generalised data storage and process execution.
\end{itemize}

A set of components plus the framework is a deployment unit which can be
embedded in another application.

At their most basic, all GATE resources are {\em Java Beans}, the Java
platform's model of software components. Beans are simply Java classes that
obey certain interface conventions:
\begin{itemize}
\item
beans must have no-argument constructors.
%
\item
beans have \emph{properties}, defined by pairs of methods named by the
convention \texttt{set\emph{Prop}} and \texttt{get\emph{Prop}}.
\end{itemize}
%

GATE uses Java Beans conventions to construct and configure resources at
runtime, and defines interfaces that different component types must implement.

%%%%%%%%%%%%%%%%%%%%%%%%%%%%%%%%%%%%%%%%%%%%%%%%%%%%%%%%%%%%%%%%%%%%%%%%%%%%%
\sect[sec:creole-model:lifecycle]{The Lifecycle of a CREOLE Resource}
%%%%%%%%%%%%%%%%%%%%%%%%%%%%%%%%%%%%%%%%%%%%%%%%%%%%%%%%%%%%%%%%%%%%%%%%%%%%%

CREOLE resources exhibit a variety of forms depending on the perspective they are
viewed from. Their implementation is as a Java class plus an XML metadata file
living at the same URL. When using GATE Developer, resources can be loaded and
viewed via the resources tree (left pane) and the `create resource' mechanism.
When programming with GATE Embedded, they are Java objects that are obtained by
making calls to GATE's {\tt Factory} class. These various incarnations are the
phases of a CREOLE resource's `lifecycle'. Depending on what sort of task you are
using GATE for, you may use resources in any or all of these phases. For example,
you may only be interested in getting a graphical view of what GATE's ANNIE
Information Extraction system (see \Chapthing\ \ref{chap:annie}) does; in this
case you will use GATE Developer to load the ANNIE resources, and load a
document, and create an ANNIE application and run it on the document. If, on the
other hand, you want to create your own resources, or modify the Java code of an
existing resource (as opposed to just modifying its grammar, for example), you
will need to deal with all the lifecycle phases.

The various phases may be summarised as:
%
\begin{description}
%
\item[Creating a new resource from scratch (bootstrapping).]
To create the binary image of a resource (a Java class in a JAR file),
and the XML file that describes the resource to GATE, you need to
create the appropriate {\tt .java} file(s), compile them and package
them as a {\tt .jar}. GATE provides a Maven archetype to
start this process -- see
Section \ref{sec:api:bootstrap}. Alternatively you can simply copy
code from an existing resource.
%
\item[Instantiating a resource in GATE Embedded.]
To create a resource in your own Java code, use GATE's {\tt Factory} class
(this takes care of parameterising the resource, restoring it from a database
where appropriate, etc. etc.). Section \ref{sec:api:factory} describes how
to do this.
%
\item[Loading a resource into GATE Developer.]
To load a resource into GATE Developer, use the various `New
... resource' options from the {\tt File} menu and elsewhere. See
Section
\ref{sec:developer:gui}.
%
\item[Resource configuration and implementation.]
GATE's Maven archetype will create an empty resource that does nothing. In
order to achieve the behaviour you require, you'll need to change the
Java code and its configuration annotations.
See section \ref{sec:creole-model:config} for more details.
%
\end{description}
%
%More details of the specifics of tasks related to these phases
%are available in \chapthing\ \ref{chap:howto}.




%%%%%%%%%%%%%%%%%%%%%%%%%%%%%%%%%%%%%%%%%%%%%%%%%%%%%%%%%%%%%%%%%%%%%%%%%%%%%
\sect[sec:creole-model:applications]{Processing Resources and Applications}
%%%%%%%%%%%%%%%%%%%%%%%%%%%%%%%%%%%%%%%%%%%%%%%%%%%%%%%%%%%%%%%%%%%%%%%%%%%%%

\mbox{ }

PRs can be combined into {\em applications}. Applications model a control
strategy for the execution of PRs. In GATE, applications are called
`controllers' accordingly.

Currently the main application types provided by GATE implement sequential or
``pipeline'' control flow. There are two main types of pipeline:
%
\begin{description}
%
\item[Simple pipelines] simply group a set of PRs together in order and
execute them in turn. The implementing class is called {\tt
SerialController}.
%
\item[Corpus pipelines] are specific for LanguageAnalysers -- PRs that are
applied to documents and corpora. A corpus pipeline opens each document in
the corpus in turn, sets that document as a runtime parameter on each PR,
runs all the PRs on the corpus, then closes the document.
The implementing class is called {\tt SerialAnalyserController}.
\end{description}

Conditional versions of these controllers are also available. These allow
processing resources to be run conditionally on document features. 
See Section \ref{sec:developer:cond} for how to use these.  If more flexibility
is required, the Groovy plugin provides a \emph{scriptable} controller (see
section~\ref{sec:api:groovy:controller}) whose execution strategy is specified
using the Groovy programming language.

Controllers are themselves PRs -- in particular a simple pipeline is a standard
PR and a corpus pipeline is a LanguageAnalyser -- so one pipeline can be nested
in another.  This is particularly useful with conditional controllers to group
together a set of PRs that can all be turned on or off as a group.

There is also a real-time version of the corpus pipeline. When creating such
a controller, a {\tt timeout} parameter needs to be set which determines the
maximum amount of time (in milliseconds) allowed for the processing of a
document. Documents that take longer to process, are simply ignored and the
execution moves to the next document after the timeout interval has lapsed.

All controllers have special handling for processing resources that implement
the interface \texttt{gate.creole.ControllerAwarePR}.  This interface provides
methods that are called by the controller at the start and end of the whole
application's execution -- for a corpus pipeline, this means before any
document has been processed and after all documents in the corpus have been
processed, which is useful for PRs that need to share data structures across
the whole corpus, build aggregate statistics, etc.  For full details, see the
JavaDoc documentation for \texttt{ControllerAwarePR}.
%


%%%%%%%%%%%%%%%%%%%%%%%%%%%%%%%%%%%%%%%%%%%%%%%%%%%%%%%%%%%%%%%%%%%%%%%%%%%%%
\sect[sec:creole-model:datastores]{Language Resources and Datastores}
%%%%%%%%%%%%%%%%%%%%%%%%%%%%%%%%%%%%%%%%%%%%%%%%%%%%%%%%%%%%%%%%%%%%%%%%%%%%%

\mbox{ }

Language Resources can be stored in Datastores. Datastores are an abstract
model of disk-based persistence, which can be implemented by various types of
storage mechanism. Here are the types implemented:
%
\begin{description}
%
\item[Serial Datastores] are based on Java's serialisation system, and store
data directly into files and directories.
%
\item[Lucene Datastores] is a full-featured annotation indexing and
retrieval system. It is provided as part of an extension of the Serial
Datastores. See Section \ref{chap:annic} for more details.
%
%\item[Oracle Datastores] store data into an Oracle RDBMS.
%For details of how to set up an Oracle DB for GATE, see
%\htlinkplain{http://gate.ac.uk/gate/doc/persistence.pdf}.
%
%\item[PostgreSQL Datastores] store data into a PostgreSQL RDBMS.
%For details of how to set up a PostgreSQL DB for GATE, see 
%\htlinkplain{http://gate.ac.uk/gate/doc/persistence.pdf}.
\end{description}

%%%%%%%%%%%%%%%%%%%%%%%%%%%%%%%%%%%%%%%%%%%%%%%%%%%%%%%%%%%%%%%%%%%%%%%%%%%%%
\sect[sec:creole-model:builtins]{Built-in CREOLE Resources}
%%%%%%%%%%%%%%%%%%%%%%%%%%%%%%%%%%%%%%%%%%%%%%%%%%%%%%%%%%%%%%%%%%%%%%%%%%%%%

\mbox{ }

GATE comes with various built-in components:
%
\begin{itemize}
%
\item
Language Resources modelling Documents and Corpora, and various types of
Annotation Schema -- see \Chapthing\ \ref{chap:corpora}.
%
\item
Processing Resources that are part of the ANNIE system -- see \Chapthing\
\ref{chap:annie}.
%
%\item
%Visual Resources for viewing and editing corpora, annotations, etc.
% -- see \chapthing\ \ref{chap:gazetteers}.
%
\item 
Gazetteers -- see \Chapthing\
\ref{chap:gazetteers}.
\item
Ontologies -- see \Chapthing\
\ref{chap:ontologies}.
\item
Machine Learning resources -- see \Chapthing\
\ref{chap:ml}.
\item
Alignment tools -- see \Chapthing\
\ref{chap:alignment}.
\item
Parsers and taggers -- see \Chapthing\
\ref{chap:parsers}.
\item
Other miscellaneous resources -- see \Chapthing\ \ref{chap:misc-creole}.
\end{itemize}

%%%%%%%%%%%%%%%%%%%%%%%%%%%%%%%%%%%%%%%%%%%%%%%%%%%%%%%%%%%%%%%%%%%%%%%%%%%%%
\sect[sec:creole-model:config]{CREOLE Resource Configuration}
%%%%%%%%%%%%%%%%%%%%%%%%%%%%%%%%%%%%%%%%%%%%%%%%%%%%%%%%%%%%%%%%%%%%%%%%%%%%%

This section describes how to supply GATE with the configuration data it needs
about a resource, such as what its parameters are, how to display it if it has
a visualisation, etc.  Several GATE resources can be grouped into a single
\emph{plugin}, which is a directory or JAR file containing an XML configuration file called
{\tt creole.xml} at its root.  The {\tt creole.xml} file provides metadata
about the plugin as a whole, the configuration for individual resource classes
is given directly in the Java source file using Java annotations.

A {\tt creole.xml} file has a root element \verb|<CREOLE-DIRECTORY>| which
supports several optional attribute:

\begin{description}
%\item[{\tt ID}:] A string that uniquely identifies this plugin. This should
%be formatted in a similar way to fully specified Java class names. The class portion
%(i.e. everything after the last dot) will be used as the name of the plugin in the
%GUI. For example, the obsolete RASP plugin could have the ID gate.obsolete.RASP.
%Note that unlike Java class names the plugin name can contain spaces for the purpose
%of presentation.
\item[{\tt NAME}:] The name of the plugin. Used in the GUI to help identify the
plugin in a nicer way than the direcory or artifact name.
\item[{\tt VERSION}:] The version number of the plugin. For example, 3, 3.1,
3.11, 3.12-SNAPSHOT etc.
\item[{\tt DESCRIPTION}:] A short description of the resources provided by the plugin.
Note that there is really only space for a single sentence in the GUI.
%\item[{\tt HELPURL}:] The URL of a web page giving more details about this plugin.
\item[{\tt GATE-MIN}:] The  earliest version of GATE that this plugin is compatible
with. This should be in the same format as the version shown in the GATE titlebar, i.e.
8.5 or 8.6-beta1. Do not include the build number information.
%\item[{\tt GATE-MAX}:] The last version of GATE which the plugin is compatible with.
%This should be in the same format as GATE-MIN.
\end{description}

Currently all these attributes are optional, as in most cases the information can
be pulled from other elements of the plugin metadata; for example, plugins distributued
via Maven will use information from the \verb|pom.xml| if not specified.

For many simple single-JAR plugins the {\tt creole.xml} file need have no other
content -- just an empty \verb|<CREOLE-DIRECTORY />| element -- but there are
certain child elements that are used in some types of plugin.

Directory-style plugins need at least one \verb|<JAR>| child element to tell
GATE where to find the classes that implement the plugin's resources.  Each
\verb|<JAR>| element contains a path to a JAR file, which is resolved relative
to the location of the {\tt creole.xml}, for example:
\begin{small}\begin{verbatim}
<CREOLE-DIRECTORY>
  <JAR SCAN="true">myPlugin.jar</JAR>
  <JAR>lib/thirdPartyLib.jar</JAR>
</CREOLE-DIRECTORY>
\end{verbatim}\end{small}

JAR files that contain resource classes must be specified with
\verb|SCAN="true"|, which tells GATE to scan the JAR contents to discover
resource classes annotated with \verb|@CreoleResource| (see below).  Other JAR
files required by the plugin can be specified using other \verb|<JAR>| elements
without \verb|SCAN="true"|.

Plugins can depend on other plugins, for example if a plugin defines a PR which
internally makes use of a JAPE transducer then that plugin would declare that
it depends on ANNIE (the standard plugin that defines the JAPE transducer PR).
This is done with a \verb|<REQUIRES>| element.  To depend on a single-JAR
plugin from a Maven repository, use an empty element with attributes
\verb|GROUP|, \verb|ARTIFACT| and \verb|VERSION|, for example
\begin{small}\begin{verbatim}
<CREOLE-DIRECTORY>
  <REQUIRES GROUP="uk.ac.gate.plugins"
            ARTIFACT="annie"
            VERSION="8.5" />
</CREOLE-DIRECTORY>
\end{verbatim}\end{small}

Directory-style plugins can also depend on other directory-style plugins using
a relative path (e.g. \verb|<REQUIRES>../other-plugin</REQUIRES>|, but this is
generally discouraged -- if your plugin is likely to be required as a
dependency of other plugins then it is better converted to the single JAR Maven
style so the dependency can be handled via group/artifact/version co-ordinates.

You may see old plugins with other elements such as \verb|<RESOURCE>|, this is
the older style of configuration in XML, which is now deprecated in favour of
the annotations described below.

%%%%%%%%%%%%%%%%%%%%%%%%%%%%%%%%%%%%%%%%%%%%%%%%%%%%%%%%%%%%%%%%%%%%%%%%%%%%%
\subsect[sec:creole-model:config:annotations]{Configuring Resources using Annotations}
%%%%%%%%%%%%%%%%%%%%%%%%%%%%%%%%%%%%%%%%%%%%%%%%%%%%%%%%%%%%%%%%%%%%%%%%%%%%%

The configuration of the resources within a plugin is handled using Java
annotation types to embed the configuration data directly in the Java source
code.  \verb|@CreoleResource| is used to mark a class as a GATE resource, and
parameter information is provided through annotations on the JavaBean {\tt set}
methods.  At runtime these annotations are read and used to construct the
resource data that is registered with the CREOLE register.  The metadata
annotation types are all marked \verb|@Documented| so the CREOLE configuration
data will be visible in the generated JavaDoc documentation.

For more detailed information, see the
JavaDoc documentation for {\tt gate.creole.metadata}.

%%%%%%%%%%%%%%%%%%%%%%%%%%%%%%%%%%%%%%%%%%%%%%%%%%%%%%%%%%%%%%%%%%%%%%%%%%%%%
\subsubsect{Basic Resource-Level Data}
%%%%%%%%%%%%%%%%%%%%%%%%%%%%%%%%%%%%%%%%%%%%%%%%%%%%%%%%%%%%%%%%%%%%%%%%%%%%%

To mark a class as a CREOLE resource, simply use the \verb|@CreoleResource|
annotation (in the {\tt gate.creole.metadata} package), for example:

\begin{lstlisting}
import gate.creole.AbstractLanguageAnalyser;
import gate.creole.metadata.*;

@CreoleResource(name = "GATE Tokeniser",
                comment = "Splits text into tokens and spaces")
public class Tokeniser extends AbstractLanguageAnalyser {
  ...
}
\end{lstlisting}

The \verb|@CreoleResource| annotation provides slots for various configuration
values:
\begin{description}
\item[name] (String) the name of the resource, as it will appear in the `New'
  menu in GATE Developer.  If omitted, defaults to the bare name of the resource
  class (without a package name).
\item[comment] (String) a descriptive comment about the resource, which will
  appear as the tooltip when hovering over an instance of this
  resource in the resources tree in GATE Developer.  If omitted, no
  comment is used.
\item[helpURL] (String) a URL to a help document on the web for this
  resource. It is used in the help browser inside GATE Developer.
\item[isPrivate] (boolean) should this resource type be hidden from the GATE Developer GUI, so
  it does not appear in the `New' menus?  If omitted, defaults to
  false (i.e.  not hidden).
\item[icon] (String) the icon to use to represent the resource in GATE Developer.
  If omitted, a generic language resource or processing resource icon
  is used. The value of this element can be:
  \begin{itemize}
  \item a plain name such as ``Application'', which is prepended with the
    package name {\tt gate.resources.img.svg.} and the suffix ``Icon'', which is
    assumed to be a Java class implementing {\tt javax.swing.Icon}.  GATE
    provides a collection of these icon classes which are generated from SVG
    files and are fully scalable for high-DPI monitors.
  \item a path to an image file inside the plugin's JAR, starting with a
    forward slash, e.g. \verb|/myplugin/images/icon.png|
  \end{itemize}
\item[interfaceName] (String) the interface type implemented by this resource,
  for example a new type of document would specify \verb|"gate.Document"| here.
\item[tool] (boolean) is this resource type a tool?  The ``tool'' flag
  identifies things like resource helpers and resources that contribute items
  to the tools menu in GATe Developer.
\item[autoInstances] (array of {\tt @AutoInstance} annotations) definitions for
  any instances of this resource that should be created automatically when the
  plugin is loaded.  If omitted, no auto-instances are created by default.
  Auto-instances are useful for things like document formats and tools which
  contribute behaviour to other GATE resources, and which should be available
  by default whenever the plugin is loaded.
\end{description}

For visual resources only, the following elements are also available:
\begin{description}
\item[guiType] (GuiType enum) the type of GUI this resource defines.
  The options are LARGE (the VR should appear in the main right-hand panel of
  the GUI) or SMALL (the VR should appear in the bottom left hand corner
  below the resources tree).
\item[resourceDisplayed] (String) the class name of the resource type that this
  VR displays, e.g. \verb|"gate.Corpus"|. Any resource whose type is assignable
  to this type will be displayed with this viewer, so for example a VR that can
  display all types of document would specify \verb|gate.Document|, whereas a
  VR that can only display the default GATE document implementation would
  specify \verb|gate.corpora.DocumentImpl|.
\item[mainViewer] (boolean) is this VR the `most important' viewer for its
  displayed resource type? If there are several different viewers that are all
  applicable to a particular resource type, the {\tt mainViewer} hint helps
  GATE Developer decide which one should be initially visible as the selected
  tab.
\end{description}

For annotation viewers, you should specify an
\verb|annotationTypeDisplayed| element giving the annotation type that the
viewer can display (e.g. {\tt Sentence}).

%%%%%%%%%%%%%%%%%%%%%%%%%%%%%%%%%%%%%%%%%%%%%%%%%%%%%%%%%%%%%%%%%%%%%%%%%%%%%
\subsubsect{Resource Parameters}
%%%%%%%%%%%%%%%%%%%%%%%%%%%%%%%%%%%%%%%%%%%%%%%%%%%%%%%%%%%%%%%%%%%%%%%%%%%%%

Parameters are declared by placing annotations on their JavaBean {\tt set}
methods.  To mark a setter method as a parameter, use the
\verb|@CreoleParameter| annotation, for example:

\begin{small}\begin{verbatim}
  @CreoleParameter(comment = "The location of the list of abbreviations")
  public void setAbbrListUrl(URL listUrl) {
    ...
\end{verbatim}\end{small}

GATE will infer the parameter's name from the name of the JavaBean property in
the usual way (i.e. strip off the leading {\tt set} and convert the following
character to lower case, so in this example the name is {\tt abbrListUrl}).
The parameter name is \emph{not} taken from the name of the method parameter.
The parameter's type is inferred from the type of the method parameter
({\tt java.net.URL} in this case).

The annotation elements of \verb|@CreoleParameter| are as follows:
\begin{description}
\item[comment] (String) an optional descriptive comment about the parameter.
\item[defaultValue] (String) the optional default value for this parameter.
  The value is specified as a string but is converted to the relevant type by
  GATE according to the conversions described below.
\item[suffixes] (String) for parameters of type URL or ResourceReference, a
  semicolon-separated list of default file suffixes that this parameter accepts.
\item[collectionElementType] (Class) for {\tt Collection}-valued parameters,
  the type of the elements in the collection.  This can usually be inferred
  from the generic type information, for example
  \verb|public void setIndices(List<Integer> indices)|, but must be specified
  if the {\tt set} method's parameter has a raw (non-parameterized) type.
\end{description}

Parameter default values must be specified as strings, but parameters can be of
any type and GATE applies the following rules to convert the default string
into an appropriate value for the parameter type:

\begin{description}
\item[String] if the parameter is of type String the default value is used
  directly
\item[Primitive wrapper types e.g. Integer] the string is passed to the
  relevant {\tt valueOf} method
\item[{\tt enum} types] the value is passed to {\tt Enum.valueOf}
\item[{\tt java.net.URL} or {\tt gate.creole.ResourceReference}] the string is
  parsed as a URI, and if the URI is relative then it is resolved against the
  plugin (for directory-style plugins this means against the location of {\tt
  creole.xml} and for Maven plugins it is the root of the plugin JAR file)
\item[collection types (Set, List, etc.)] the string is treated as a
  semicolon-separated list of values, and each value is converted to the
  collection's element type following these same rules.
\item[{\tt gate.FeatureMap}] the string is parsed as
  ``feature1=value1;feature2=value2'' etc. (a semicolon-separated list of
  ``name=value'' pairs)
\item[any other {\tt java.*} type] if the type has a constructor taking a
  String then that constructor is called with the default string as its
  parameter.
\end{description}

If there is no default specified, the default value is {\tt null}.

Mutually-exclusive parameters are handled by adding a {\tt disjunction="{\it
label}"} and {\tt priority={\it n}} to the \verb|@CreoleParameter| annotation
-- all parameters that share the same label are grouped in the same
disjunction, and will be offered in order of priority.  The parameter with the
smallest priority value will be the one listed first, and thus the one that is
offered initially when creating a resource of this type in GATE Developer.  For
example, the following is a simplified extract from
{\tt gate.corpora.DocumentImpl}:

\begin{lstlisting}
@CreoleParameter(disjunction="src", priority=1)
public void setSourceUrl(URL src) { /* */ }

@CreoleParameter(disjunction="src", priority=2)
public void setStringContent(String content) { /* */ }
\end{lstlisting}

This declares the parameters ``stringContent'' and ``sourceUrl'' as
mutually-exclusive, and when creating an instance of this resource in GATE
Developer the parameter that will be shown initially is sourceUrl.  To set
stringContent instead the user must select it from the drop-down list.
Parameters with the same declared priority value will appear next to each other
in the list, but their relative ordering is not specified.  Parameters with no
explicit priority are always listed {\it after} those that do specify a
priority.

Optional and runtime parameters are marked using extra annotations, for example:
\begin{lstlisting}
  @Optional
  @RunTime
  @CreoleParameter
  public void setAnnotationSetName(String asName) {
    ...
\end{lstlisting}

Runtime parameters apply only to Processing Resources, and are parameters that
are not used when the resource is initialised but instead only when it is
executed.  An ``optional'' parameter is one that does not have to be set
before creating or executing the resource.

%%%%%%%%%%%%%%%%%%%%%%%%%%%%%%%%%%%%%%%%%%%%%%%%%%%%%%%%%%%%%%%%%%%%%%%%%%%%%
\subsubsect{Inheritance}
%%%%%%%%%%%%%%%%%%%%%%%%%%%%%%%%%%%%%%%%%%%%%%%%%%%%%%%%%%%%%%%%%%%%%%%%%%%%%

A resource
will inherit any configuration data that was not explicitly specified
from annotations on its parent class and on any interfaces it
implements.  Specifically, if you do not specify a comment,
interfaceName, icon, annotationTypeDisplayed or the GUI-related
elements (guiType and resourceDisplayed) on
your \verb|@CreoleResource| annotation then GATE will look up the
class tree for other \verb|@CreoleResource| annotations, first on the
superclass, its superclass, etc., then at any implemented interfaces,
and use the first value it finds.  This is useful if you are defining
a family of related resources that inherit from a common base class.

The resource name and the {\tt isPrivate} and {\tt mainViewer} flags are
\emph{not} inherited.

Parameter definitions are inherited in a similar way.  For
example, the {\tt gate.LanguageAnalyser} interface provides two parameter
definitions via annotated {\tt set} methods, for the {\tt corpus} and
{\tt document} parameters.  Any \verb|@CreoleResource| annotated class that
implements {\tt LanguageAnalyser}, directly or indirectly, will get these
parameters automatically.

Of course, there are some cases where this behaviour is not desirable, for
example if a subclass calculates a value for a superclass parameter rather than
having the user set it directly.  In this case you can hide the parameter by
overriding the {\tt set} method in the subclass and using a marker annotation:
\begin{lstlisting}
  @HiddenCreoleParameter
  public void setSomeParam(String someParam) {
    super.setSomeParam(someParam);
  }
\end{lstlisting}

The overriding method will typically just call the superclass one, as its only
purpose is to provide a place to put the \verb|@HiddenCreoleParameter|
annotation.

Alternatively, you may want to override some of the configuration for a
parameter but inherit the rest from the superclass.  Again, this is handled by
trivially overriding the {\tt set} method and re-annotating it:
\begin{lstlisting}
  // superclass
  @CreoleParameter(comment = "Location of the grammar file",
                   suffixes = "jape")
  public void setGrammarUrl(URL grammarLocation) {
    ...
  }

  @Optional
  @RunTime
  @CreoleParameter(comment = "Feature to set on success")
  public void setSuccessFeature(String name) {
    ...
  }
\end{lstlisting}
\begin{lstlisting}
  //-----------------------------------
  // subclass
  
  // override the default value, inherit everything else
  @CreoleParameter(defaultValue = "resources/defaultGrammar.jape")
  public void setGrammarUrl(URL url) {
    super.setGrammarUrl(url);
  }

  // we want the parameter to be required in the subclass
  @Optional(false)
  @CreoleParameter
  public void setSuccessFeature(String name) {
    super.setSuccessFeature(name);
  }
\end{lstlisting}

Note that for backwards compatibility, data is only inherited from superclass
annotations if the subclass is itself annotated with \verb|@CreoleResource|.

%%%%%%%%%%%%%%%%%%%%%%%%%%%%%%%%%%%%%%%%%%%%%%%%%%%%%%%%%%%%%%%%%%%%%%%%%%%%%
\subsect[sec:creole-model:config:dependencies]{Loading Third-Party Libraries in a Maven plugin}
%%%%%%%%%%%%%%%%%%%%%%%%%%%%%%%%%%%%%%%%%%%%%%%%%%%%%%%%%%%%%%%%%%%%%%%%%%%%%

A Maven plugin is distributed as a single JAR file, but if the plugin depends
on any third-party libraries these can be specified as dependencies in the
corresponding POM file in the usual Maven way as compile or runtime scoped
dependencies.

If one plugin has a \emph{compile-time} dependency on another (as opposed to
simply a runtime dependency when one plugin creates resources defined in
another) then you should specify the dependency in your POM as
\verb|<scope>provided</scope>| as well as declaring it in {\tt creole.xml} with
group/artifact/version.

%%%%%%%%%%%%%%%%%%%%%%%%%%%%%%%%%%%%%%%%%%%%%%%%%%%%%%%%%%%%%%%%%%%%%%%%%%%%%
\sect[sec:creole-model:tools]{Tools: How to Add Utilities to GATE Developer}
%%%%%%%%%%%%%%%%%%%%%%%%%%%%%%%%%%%%%%%%%%%%%%%%%%%%%%%%%%%%%%%%%%%%%%%%%%%%%

Visual Resources allow a developer to provide a GUI to interact with a
particular resource type (PR or LR), but sometimes it is useful to provide
general utilities for use in the GATE Developer GUI that are not tied to any
specific resource type.  Examples include the annotation diff tool and the
Groovy console (provided by the \verb|Groovy| plugin), both of which are
self-contained tools that display in their own top-level window.  To support
this, the CREOLE model has the concept of a {\em tool}.

A resource type is marked as a tool by setting \verb|tool = true| in the
@CreoleResource annotation.  If a resource is declared to
be a tool, and written to implement the \verb|gate.gui.ActionsPublisher|
interface, then whenever an instance of the resource is created its published
actions will be added to the ``Tools'' menu in GATE Developer.

Since the published actions of {\em every} instance of the resource will be
added to the tools menu, it is best not to use this mechanism on resource types
that can be instantiated by the user.  The ``tool'' marker is best used in
combination with the ``private'' flag (to hide the resource from the list of
available types in the GUI) and one or more hidden autoinstance definitions
to create a limited number of instances of the resource when its defining
plugin is loaded.  See the \verb|GroovySupport| resource in the \verb|Groovy|
plugin for an example of this.

\subsect[sec:creole-model:tools:menu-path]{Putting Your Tools in a Sub-Menu}

If your plugin provides a number of tools (or a number of actions from the same
tool) you may wish to organise your actions into one or more sub-menus, rather
than placing them all on the single top-level tools menu.  To do this, you need
to put a special value into the actions returned by the tool's
\lstinline!getActions()! method:
\begin{lstlisting}
action.putValue(GateConstants.MENU_PATH_KEY,
    new String[] {"Acme toolkit", "Statistics"});
\end{lstlisting}
The key must be \lstinline!GateConstants.MENU_PATH_KEY! and the value must be
an array of strings.  Each string in the array represents the name of one level
of sub-menus.  Thus in the example above the action would be placed under
``Tools $\rightarrow$ Acme toolkit $\rightarrow$ Statistics''.  If no
\lstinline!MENU_PATH_KEY! value is provided the action will be placed directly
on the Tools menu.

\subsect[sec:creole-model:tools:resourcehelpers]{Adding Tools To Existing
Resource Types}

While Visual Resources (VR) allow you to add new features to a particular
resource they have a number of shortcomings. Firstly not every new feature
will require a full VR; often a new entry on the resources right-click
menu will suffice. More importantly new feautres added via a VR are only
available while the VR is open. A Resource Helper is a form of Tool, as
above, which can add new menu options to any existing resource type without
requiring a VR.

A Resource Helper is defined in the same way as a Tool (by setting the
\verb|tool = true| feature of the @CreoleResource annotation and loaded via an
autoinstance definition) but must also extend the \verb|gate.gui.ResourceHelper|
class. A Resource Helper can then return a set of actions for a given resource
which will be added to its right-click menu. See the \verb|FastInfosetExporter|
resource in the ``Format: FastInfoset'' plugin for an example of how this works.

A Resource Helper may also make new API calls accessable to allow similar
functionality to be made available to GATE Embedded, see Section
\ref{sec:api:resourcehelpers} for more details on how this works.
